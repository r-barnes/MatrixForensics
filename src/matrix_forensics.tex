\documentclass{book}

%Post to: https://stats.stackexchange.com/questions/21346/reference-book-for-linear-algebra-applied-to-statistics

\usepackage[top=1in, bottom=1.25in, left=1.25in, right=1.25in]{geometry}

\usepackage{amsfonts, amsmath}
\usepackage{commath}
\usepackage[yyyymmdd,hhmmss]{datetime}
\usepackage{graphbox}
\usepackage[hidelinks]{hyperref}
\usepackage{marginnote}
\usepackage{mathtools}
\usepackage{parskip}
\usepackage{titlesec}
\usepackage{xcolor}
\usepackage{optidef}

\usepackage{cellspace}%
\setlength\cellspacetoplimit{3pt}
\setlength\cellspacebottomlimit{3pt}

\usepackage{makeidx}
\makeindex

\usepackage[numbers,sort&compress]{natbib}
\bibliographystyle{unsrtnat}

%Make equations be numbered continuously through book
\usepackage{chngcntr}
\counterwithout{equation}{chapter}

\renewcommand{\sectionautorefname}{\textsection}
\renewcommand{\subsectionautorefname}{\textsection}
\renewcommand{\subsubsectionautorefname}{\textsection}

% This file contains all of the mathematical commands used in the book

\renewcommand*{\pd}[3][]{\ensuremath{\frac{\partial^{#1} #2}{\partial #3}}}

\newcommand{\mA}{\mathbf{A}}
\newcommand{\mB}{\mathbf{B}}
\newcommand{\mC}{\mathbf{C}}
\newcommand{\mD}{\mathbf{D}}
\newcommand{\mE}{\mathbf{E}}
\newcommand{\mF}{\mathbf{F}}
\newcommand{\mH}{\mathbf{H}}
\newcommand{\mI}{\mathbf{I}}
\newcommand{\mJ}{\mathbf{J}}
\newcommand{\mL}{\mathbf{L}}
\newcommand{\mM}{\mathbf{M}}
\newcommand{\mP}{\mathbf{P}}
\newcommand{\mQ}{\mathbf{Q}}
\newcommand{\mR}{\mathbf{R}}
\newcommand{\mS}{\mathbf{S}}
\newcommand{\mU}{\mathbf{U}}
\newcommand{\mV}{\mathbf{V}}
\newcommand{\mX}{\mathbf{X}}
\newcommand{\mY}{\mathbf{Y}}

\newcommand{\mAi}{\mathbf{A}^{-1}}
\newcommand{\mBi}{\mathbf{B}^{-1}}
\newcommand{\mCi}{\mathbf{C}^{-1}}
\newcommand{\mPi}{\mathbf{P}^{-1}}
\newcommand{\mRi}{\mathbf{R}^{-1}}
\newcommand{\mXi}{\mathbf{X}^{-1}}
\newcommand{\mYi}{\mathbf{Y}^{-1}}

\newcommand{\mXp}{\mathbf{X}^{+}}


%%%%% TRANSPOSES
\newcommand{\T}{^\mathsf{T}}
\newcommand{\mAT}{\mathbf{A}^{\mathsf{T}}}
\newcommand{\mBT}{\mathbf{A}^{\mathsf{T}}}
\newcommand{\mCT}{\mathbf{A}^{\mathsf{T}}}
\newcommand{\mDT}{\mathbf{A}^{\mathsf{T}}}
\newcommand{\mET}{\mathbf{A}^{\mathsf{T}}}
\newcommand{\mXT}{\mathbf{X}^{\mathsf{T}}}

\newcommand{\mXiT}{\mathbf{X}^{-\mathsf{T}}}

\newcommand{\va}{\mathbf{a}}
\newcommand{\vb}{\mathbf{b}}
\newcommand{\vc}{\mathbf{c}}
\newcommand{\vd}{\mathbf{d}}
\newcommand{\ve}{\mathbf{e}}
\newcommand{\vf}{\mathbf{f}}
\newcommand{\vg}{\mathbf{g}}
\newcommand{\vp}{\mathbf{p}}
\newcommand{\vq}{\mathbf{q}}
\newcommand{\vu}{\mathbf{u}}
\newcommand{\vv}{\mathbf{v}}
\newcommand{\vw}{\mathbf{w}}
\newcommand{\vx}{\mathbf{x}}
\newcommand{\vy}{\mathbf{y}}
\newcommand{\vz}{\mathbf{z}}

\newcommand{\vaT}{\mathbf{a}^\mathsf{T}}
\newcommand{\vbT}{\mathbf{b}^\mathsf{T}}
\newcommand{\vuT}{\mathbf{u}^\mathsf{T}}
\newcommand{\vvT}{\mathbf{v}^\mathsf{T}}
\newcommand{\vxT}{\mathbf{x}^\mathsf{T}}

\newcommand{\vzero}{\mathbf{0}}
\DeclareMathOperator{\diag}{diag}
\DeclareMathOperator{\eig}{eig}
\DeclareMathOperator{\trace}{tr}
\DeclareMathOperator{\rank}{rank}
\DeclareMathOperator{\nnz}{nnz}
\newcommand{\sPSD}{\mathbb{S}^n_+}
\newcommand{\sC}{\mathbb{C}}
\newcommand{\sCmn}{\mathbb{C}^{m,n}}
\newcommand{\sCnn}{\mathbb{C}^{n,n}}
\newcommand{\sR}{\mathbb{R}}
\newcommand{\sRm}{\mathbb{R}^{m}}
\newcommand{\sRn}{\mathbb{R}^{n}}
\newcommand{\sRp}{\mathbb{R}^{p}}
\newcommand{\sRkk}{\mathbb{R}^{k,k}}
\newcommand{\sRkn}{\mathbb{R}^{k,n}}
\newcommand{\sRnm}{\mathbb{R}^{n,m}}
\newcommand{\sRmn}{\mathbb{R}^{m,n}}
\newcommand{\sRnn}{\mathbb{R}^{n,n}}
\newcommand{\sRnk}{\mathbb{R}^{n,k}}
\newcommand{\sRnp}{\mathbb{R}^{n,p}}
\newcommand{\sRnr}{\mathbb{R}^{n,r}}
\newcommand{\sRmm}{\mathbb{R}^{m,m}}
\newcommand{\sSn}{\mathbb{S}^{n}}
\newcommand{\ispsd}{\succeq}
\newcommand{\ispd}{\succ}
\newcommand{\pinv}{\!^+}
\newcommand{\ns}{\mathcal{N}}
\newcommand{\range}{\mathcal{R}}
\newcommand{\bs}{\setminus}
\newcommand{\kp}{\otimes}       %Kronecker product
\newcommand{\hp}{\circ}         %Hadamard product
\newcommand{\grad}{\nabla}      %Gradient operator

\hypersetup{
  pdfauthor={Richard Barnes (ORCID: 0000-0002-0204-6040)},%
  pdftitle={Matrix Forensics},%
%            pdfsubject={Whatever},%
  pdfkeywords = {matrix algebra, matrix relations, matrix identities, linear algebra},%
  pdfproducer = {LaTeX},%
  pdfcreator  = {pdfLaTeX}
}


\usepackage{fancyhdr}
% \renewcommand{\chaptermark}[1]{\markboth{#1}{#1}}
\setlength{\headheight}{15.2pt}
\pagestyle{fancy}

\lhead[\thepage]{\leftmark}
% \chead[]{<odd output>}
\rhead[\leftmark]{\thepage}

\renewcommand{\footrulewidth}{0.4pt}% default is 0pt
\lfoot[\footnotesize{Richard Barnes. Matrix Forensics. \today-\currenttime. \href{https://github.com/r-barnes/MatrixForensics}{github.com/r-barnes/MatrixForensics}}. \input{/tmp/matrix_forensics_version.info}\!\!.]{\footnotesize{Richard Barnes. Matrix Forensics. \today-\currenttime. \href{https://github.com/r-barnes/MatrixForensics}{github.com/r-barnes/MatrixForensics}}. \input{/tmp/matrix_forensics_version.info}\!\!.} %  [<even output>]{<odd output>}
\cfoot[]{}
\rfoot[]{}


\newcommand{\eqcite}[1]{\marginnote{\citep{#1}}}

%Adjust chapter formatting
\newcommand{\hsp}{\hspace{20pt}}
\definecolor{gray75}{gray}{0.75}
\titleformat{\chapter}[hang]{\Huge\bfseries}{\thechapter\hsp\textcolor{gray75}{$|$}\hsp}{0pt}{\Huge\bfseries}
\titlespacing*{\chapter}{0pt}{0pt}{20pt} %? BEFORE AFTER

%Ensure chapters start on the same page
\usepackage{etoolbox}
\makeatletter
\patchcmd{\chapter}{\if@openright\cleardoublepage\else\clearpage\fi}{\clearpage}{}{}
\makeatother


\begin{document}

\begin{titlepage} % Suppresses displaying the page number on the title page and the subsequent page counts as page 1
  \raggedleft % Right align the title page
  
  \rule{1pt}{\textheight} % Vertical line
  \hspace{0.05\textwidth} % Whitespace between the vertical line and title page text
  \parbox[b]{0.75\textwidth}{ % Paragraph box for holding the title page text, adjust the width to move the title page left or right on the page
    
    {\Huge\bfseries Matrix Forensics \\[\baselineskip]} % Title
    {\large\textit{A brief guide to matrix math \\ and its efficient implementation}}\\[4\baselineskip] % Subtitle or further description
    {\Large\textsc{richard barnes}} % Author name, lower case for consistent small caps
    \\[4\baselineskip]
    \immediate\write18{ git rev-parse HEAD > /tmp/matrix_forensics_version.info }

    Git Hash: \input{/tmp/matrix_forensics_version.info} \\
    Compiled on: \today\ at \currenttime
    \\[2\baselineskip]
    \href{https://github.com/r-barnes/MatrixForensics}{github.com/r-barnes/MatrixForensics}
    
    \vspace{0.4\textheight} % Whitespace between the title block and the publisher
    
    %{\noindent The Publisher~~\plogo}\\[\baselineskip] % Publisher and logo
  }

\end{titlepage}


\tableofcontents

\chapter{Introduction}

\textbf{Goals:}
TODO



\textbf{Contributing:}
Please contribute on Github at \url{https://github.com/r-barnes/MatrixForensics} either by opening an issue or making a pull request. If you are not comfortable with this, please send your contribution to \url{rijard.barnes@gmail.com}.


\textbf{Contributors:}
Richard Barnes

\textbf{Funding:}
TODO

\chapter{Nomenclature}

\begin{tabular}{cl}
$\mA$                   & Matrix.                                                   \\
$\va$                   & (Column) vector.                                          \\
$a$                     & Scalar.                                                   \\
& \\
$\mA_{ij}$              & Matrix indexed. Returns $i$th row and $j$th column.       \\
$\mA\circ \mB$          & Hadamard (element-wise) product of matrices A and B.      \\
$\ns(\mA)$              & Nullspace of the matrix $\mA$.                            \\
$\range(\mA)$           & Range of the matrix $\mA$.                                \\
$\det(\mA)$             & Determinant of the matrix $\mA$.                          \\
$\eig(\mA)$             & Eigenvalues of the matrix $\mA$.                          \\
$\mA^H$                 & Conjugate transpose of the matrix $\mA$.                  \\
$\mA^T$                 & Transpose of the matrix $\mA$.                            \\
$\mA\pinv$              & Pseudoinverse of the matrix $\mA$.                        \\
$\vx\in\sRn$            & The entries of the $n$-vector $\vx$ are all real numbers. \\
$\mA\in\sRmn$           & The entries of the matrix $\mA$ with $m$ rows and $n$ columns are all real numbers. \\
$\mA\in\sSn$            & The matrix $\mA$ is symmetric and has $n$ rows and $n$ columns. \\
& \\
$\mI_n$                 & Identity matrix with $n$ rows and $n$ columns.            \\
& \\
$\{0\}$                 & The empty set
\end{tabular}

\chapter{Basics}

\section{Fundamental Theorem of Linear Algebra}

\begin{center}
\includegraphics[width=\textwidth]{imgs/fund_theorem_lin_alg1.png}
\includegraphics[width=\textwidth]{imgs/fund_theorem_lin_alg2.png}
\includegraphics[width=\textwidth]{imgs/fund_theorem_lin_alg3.png}
\includegraphics[width=\textwidth]{imgs/fund_theorem_lin_alg4.png}
\includegraphics[width=\textwidth]{imgs/fund_theorem_lin_alg5.png}
\end{center}


\section{Matrix Properties}

\begin{align}
\mA(\mB+\mC) &=   \mA\mB+\mA\mC &\textrm{(left distributivity)}   \\
(\mB+\mC)\mA &=   \mB\mA+\mC\mA &\textrm{(right distributivity)}  \\
\mA\mB       &\ne \mB\mA        &\textrm{(in general)}            \\
(\mA\mB)\mC  &=   \mA(\mB\mC)   &\textrm{(associativity)}
\end{align}

\section{Rank}

\begin{align}
\noalign{If $\mA\in\sRmn$ and $\mB\in\sRnr$, then}
\eqcite{Thome2016}
\rank(\mA)+\rank(\mB)-n\le \rank(\mA\mB)\le \min(\rank(\mA),\rank(\mB)) &&~~~~\textrm{Sylvester's Inequality} \\
\noalign{If $\mA\mB$, $\mA\mB\mC$, $\mB\mC$ are defined, then}
\eqcite{Thome2016}
\rank(\mA\mB)+\rank(\mB\mC)\le \rank(\mB)+\rank(\mA\mB\mC) && \textrm{Frobenius's inequality} \\
\noalign{If $\dim(\mA)=\dim(\mB)$, then}
\rank(\mA+\mB)\le\rank(\mA)+\rank(\mB) &&\textrm{Subadditivity}
\end{align}
If $\mA_1, \mA_2, \ldots, \mA_l$ have $n_1,n_2,\ldots,n_l$ columns, so that $\mA_1\mA_2\ldots\mA_l$ is well-defined, then
\begin{equation}
\eqcite{Thome2016}
\rank(\mA_1\mA_2\ldots\mA_l)
\ge \sum_{i=1}^{l-1}\rank(\mA_i\mA_{i+1})-\sum_{i=2}^{l-1}\rank(\mA_i)
\ge\sum_{i=1}^l\rank(\mA_i)-\sum_{i=1}^{l-1}n_i
\end{equation}

\section{Identities}
\begin{align}
\left(\sum_{i=1}^n \vz_i\right)^2 = \vz^T
\begin{bmatrix}
1      & \hdots & 1      \\
\vdots & \ddots & \vdots \\
1      & \hdots & 1
\end{bmatrix}
\vz
\end{align}

\section{Matrix Multiplication}

For $\mA\in\sR^{i,j}$ and $\mB\in\sR^{j,k}$ and $\mC\in\sR^{l,k}$
\begin{align}
[\mA\mB]_{ik} &= \sum_j \mA_{ij}\mB_{jk} \\
[\mA\mB\mC^T]_{il} &= \sum_j \mA_{ij}[\mB\mC^T]_{jl}=\sum_j \mA_{ij}\sum_k \mB_{jk}\mC_{lk}=\sum_j\sum_k \mA_{ij}\mB_{jk}\mC_{lk}
\end{align}
%TODO: Algorithms and orderings



\section{Transpose Properties}

\begin{align}
(c\mA)^T            &= c\mA^T                \\
(\mA\mB)^T          &= \mB^T\mA^T            \\
(\mA\mB\mC\ldots)^T &= \ldots\mC^T\mB^T\mA^T \\
(\mA+\mB)^T         &= \mA^T+\mB^T           \\
(\mA+\mB+\ldots)^T  &= \mA^T+\mB^T+\ldots^T  \\
(\mA^{-1})^T        &= (\mA^T)^{-1}
\end{align}

\section{Conjugate Tranpose}

\begin{align}
(\mA^H)^{-1}        &= (\mA^{-1})^H          \\
(\mA+\mB)^H         &= \mA^H+\mB^H           \\
(\mA+\mB+\ldots)^H  &= \mA^H+\mB^H+\ldots^H  \\
(\mA\mB)^H          &= \mB^H \mA^H           \\
(\mA\mB\mC\ldots)^H &= \ldots\mC^H\mB^H\mA^H
\end{align}


\section{Determinant Properties}
The determinant is only defined for square matrices; here we assume that $\mA\in\sRnn$.

\begin{align}
\det(\mI_n)        &= 1                                   \\
\det(\mA^T)        &= \det(\mA)                           \\
\det(\mA^H)        &= \det(\mA)^H                         \\
\det(\mA^{-1})     &= 1/\det(\mA)                         \\
\det(\mA\mB)       &= \det(\mB\mA)                        \\
\det(\mA\mB)       &= \det(\mA)\det(\mB)                  &\mB\in\sRnn \\
\det(c\mA)         &= c^n\det(\mA)                        \\
\det(\mA)          &= \prod \eig(\mA)                     \\
\det(\mA^n)        &= \det(\mA)^n                         \\
\det(-\mA)         &= (-1)^n\det(\mA)                     \\
\det(\mA^c)        &= \det(\mA)^c                         \\
\det(\mI+\vu \vv^T)&= 1 + \vu^T \vv                       \\
\det(\mB\mA\mB^{-1}) &= \det(\mA)                         \\
\det(\mB\mA\mB^{-1}-c\mI) &= \det(\mA-c\mI)               \\
\noalign{For n=2:}
\det(\mI+\mA)      &= 1 + \det(\mA)+\trace(\mA) \\
\det(\mA) &=\begin{vmatrix} a & b \\ c & d \end{vmatrix} = ad-bc \\
\noalign{For n=3:}
\det(\mI+\mA)      &= 1 + \det(\mA)+\trace(\mA) + \frac{1}{2}\trace(\mA)^2-\frac{1}{2}\trace(\mA^2) \\
\det(\mA) &=\begin{vmatrix} a & b & c \\ d & e & f \\ g & h & i \end{vmatrix} =
 a\begin{vmatrix} e & f \\ h & i \end{vmatrix}
-b\begin{vmatrix} d & f \\ g & i \end{vmatrix}
+c\begin{vmatrix} d & e \\ g & h \end{vmatrix} \\
\noalign{For n=4:}
\det(\mI+\mA)      &= 1 + \det(\mA)+\trace(\mA) + \frac{1}{2}\trace(\mA)^2-\frac{1}{2}\trace(\mA^2)  \\
                   &    + \frac{1}{6}\trace(\mA)^3-\frac{1}{2}\trace(\mA)\trace(\mA^2)+\frac{1}{3}\trace(\mA^3) \\
\noalign{For small $\epsilon$:}
\det(\mI+\epsilon\mA) &\approx 1 + \det(\mA) + \epsilon\trace(\mA)+\frac{1}{2}\epsilon^2\trace(\mA)^2-\frac{1}{2}\epsilon^2\trace(\mA^2) \\ %TODO: Check from MatrixCookbook
\det(\mI+\epsilon\mA) &\approx 1 + \epsilon\trace(\mA) + O(\epsilon^2) \\ %TODO: Check: From MathWorld
\noalign{Sylvester's determinant identity, for $\mA\in\sRmn, \mB\in\sRnm$}
\eqcite{Sylvester1851}
\det(\mI_m+\mA\mB) &= \det(\mI_n+\mB\mA)                    \\
\det(\mX+\mA\mB)   &= \det(\mX)\det(\mI_n + \mB\mX^{-1}\mA) \\
\noalign{If $\mA$ is triangular}
\det(\mA) &= \prod_i \mA_{i,i} = \prod_i \diag(\mA)_i \\
\noalign{If all entries of $\mA\in\sCnn$ are in the unit disk}
\det(\mA)\le n^{n/2} \eqcite{Hadamard1893} \\
\noalign{Schur's determinant identities}
\det(\mM) &= \det(\begin{bmatrix} \mA & \mB \\ \mC & \mD \end{bmatrix}) = \det(\mA) \det(\mD -\mC \mA^{-1}\mB) \\
\det(\mM) &= \det(\begin{bmatrix} \mA & \mB \\ \mC & \mD \end{bmatrix}) = \det(\mD) \det(\mA -\mB \mD^{-1}\mC) \\
\end{align}
%TODO: Matix exponential identities det(A)=log(det(exp(A)))

Geometrically, if a unit volume is acted on by $\mA$, then $|\det(\mA)|$ indicates the volume after the transformation.


\section{Trace Properties}
The Trace is only defined for square matrices.
\begin{align}
\trace(\mA)      &=\sum_i \mA_{ii}               \\
\trace(\mA)      &=\sum_i \eig(\mA)              \\
\trace(\mA+\mB)  &=\trace(\mA)+\trace(\mB)       \\
\trace(c\mA)     &=c\trace(\mA)                  \\
\trace(\mA)      &=\trace(\mA^T)                 \\
\trace(\mA\mB)   &=\trace(\mB\mA)                \\
\trace(\mA^T\mB) &=\sum_{i,j} \mA_{ij}\mB_{ij}   \\  %TODO: For real matrices only?
\trace(\mA^T\mB) &=\sum_{i,j} (\mA\circ\mB)_{ij} \\  %TODO: For real matrices only?
\va^T \va        &=\trace(\va \va^T)
\end{align}

For $\mA,\mB,\mC,\mD$ of compatible dimensions,

\begin{equation}
\trace(\mA^T\mB)=\trace(\mA\mB^T)=\trace(\mB^T\mA)=\trace(\mB\mA^T)
\end{equation}
\begin{equation}
\trace(\mA\mB\mC\mD)=\trace(\mB\mC\mD\mA)=\trace(\mC\mD\mA\mB)=\trace(\mD\mA\mB\mC)
\end{equation}
(Invariant under cyclic permutations)



\section{Inverse Properties}
The inverse of $\mA\in\sCnn$ is denoted $\mA^{-1}$ and defined such that
\begin{equation}
\mA\mA^{-1}=\mA^{-1}\mA=\mI_n
\end{equation}
where $\mI_n$ is the $n \times n$ identity matrix. $\mA$ is nonsingular if $\mA^{-1}$ exists; otherwise, $\mA$ is singular.


If individual inverses exist
\begin{equation}
(\mA\mB)^{-1}=\mB^{-1}\mA^{-1}
\end{equation}
more generally
\begin{equation}
(\mA\mB\mC\ldots)^{-1}=\ldots\mC^{-1}\mB^{-1}\mA^{-1}
\end{equation}

\begin{equation}
(\mA^{-1})^T=(\mA^T)^{-1}
\end{equation}
\begin{equation}
(\mA^H)^{-1}=(\mA^{-1})^H
\end{equation}

Hua's Identity:
\begin{align}
(\mA+\mB)^{-1} &= \mAi - (\mA+\mA\mBi\mA)^{-1} \\
(\mA-\mB)^{-1} &= \sum_{k=0}^\infty (\mAi\mB)^k\mAi \\
\end{align}




\section{Moore--Penrose PseudoInverse}
For $\mA\in\sRmn$, the Moore--Penrose pseudoinverse $\mA\pinv$ satisfies:
\begin{align}
\mA\mA\pinv\mA      &= \mA                            \\
\mA\pinv\mA\mA\pinv &= \mA\pinv                       \\
(\mA\mA\pinv)^T     &= \mA\mA\pinv\ \textrm{(symmetric)} \\
(\mA\pinv\mA)^T     &= \mA\pinv\mA\ \textrm{(symmetric)}
\end{align}
If $\mA\pinv$ exists, it is unique. For complex matrices the symmetry condition is replaced by a requirement that the matrix be Hermitian.

If $\mA\in\sCmn$, then:
\begin{align}
(\mA\pinv)\pinv     &=   \mA                  \\
(\mA^T)\pinv        &=   (\mA\pinv)^T         \\
(\mA^H)\pinv        &=   (\mA\pinv)^H         \\
(\mA^*)\pinv        &=   (\mA\pinv)^*         \\
(\mA\pinv\mA)\mA^H  &=   \mA^H                \\
(\mA\pinv\mA)\mA^T  &\ne \mA^T                \\
(c\mA)\pinv         &=   (1/c)\mA\pinv        \\
\mA\pinv            &=   (\mA^T\mA)\pinv\mA^T \\
\mA\pinv            &=   \mA^T(\mA\mA^T)\pinv \\
(\mA^T\mA)\pinv     &=   \mA\pinv(\mA^T)\pinv \\
(\mA\mA^T)\pinv     &=   (\mA^T)\pinv\mA\pinv \\
\mA\pinv            &=   (\mA^H\mA)\pinv\mA^H \\
\mA\pinv            &=   \mA^H(\mA\mA^H)\pinv \\
(\mA^H\mA)\pinv     &=   \mA\pinv(\mA^H)\pinv \\
(\mA\mA^H)\pinv     &=   (\mA^H)\pinv\mA\pinv \\
(\mA\mB)\pinv       &=   (\mA\pinv\mA\mB)\pinv(\mA\mB\mB\pinv)\pinv
\end{align}

If $\mA$ is full-rank, then:
\begin{align}
(\mA\mA\pinv)(\mA\mA\pinv) &= \mA\mA\pinv                           \\
(\mA\pinv\mA)(\mA\pinv\mA) &= \mA\pinv\mA                           \\
\trace(\mA\mA\pinv)        &= \rank(\mA\mA\pinv) \eqcite{Seber2002} \\
\trace(\mA\pinv\mA)        &= \rank(\mA\pinv\mA) \eqcite{Seber2002}
\end{align}

\subsection*{Special Properties}
\begin{itemize}
\item $\mA\pinv=\mA^{-1}$ if $\mA\in\sRnn$ and $\mA$ is square and nonsingular.
\item $\mA\pinv=(\mA^T\mA)^{-1}\mA^T$, if $\mA\in\sRmn$ is full column rank ($r=n\le m$). $\mA\pinv$ is a left inverse of $\mA$, so $\mA\pinv\mA=\mV_r\mV_r^T=\mV\mV^T=\mI_n$.
\item $\mA\pinv=\mA^T(\mA\mA^T)^{-1}$, if $\mA\in\sRmn$ is full row rank ($r=m\le n$). $\mA\pinv$ is a right inverse of $\mA$, so $\mA\mA\pinv=\mU_r\mU_r^T=\mU\mU^T=\mI_m$.
\end{itemize} %TODO: Check these


%TODO
% \subsection{Moore-Penrose Pseudoinverse}
% \begin{equation}
% \mA\pinv = \mV \mD^{-1} \mU^T
% \end{equation}
% where the foregoing comes from a singular-value decomposition and $\mD^{-1}=\diag(\frac{1}{\sigma_1},\ldots,\frac{1}{\sigma_r})$



\section{Hadamard Identities}

\begin{align}
(\mA\circ\mB)_{ij}    &= A_{ij}B_{ij}~\forall~i,j                                     \\
\mA\circ\mB           &= \mB\circ\mA                             \eqcite{million2007} \\
\mA\circ(\mB\circ\mC) &= (\mA\circ\mB)\circ\mC                                        \\
\mA\circ(\mB+\mC)     &= \mA\circ\mB+\mA\circ\mC                 \eqcite{million2007} \\
a(\mA\circ\mB)        &= (a\mA)\circ\mB =\mA\circ(a\mB)          \eqcite{million2007} \\
(\mA^T\circ\mB^T)     &= (\mA\circ\mB)^T                                              \\
(\mA^T\circ\mB^T)     &= (\mA\circ\mB)^T                                              \\
(\vx^T \mA \vx)       &= \sum_{i,j}\big((\vx \vx^T)\circ\mA\big)                      \\
\vx^T(\mA\circ\mB)\vy &= \trace((\diag(\vx)\mA)^T \mB\diag(\vy))~~~\mA,\mB\in\sRmn \eqcite{Minka2000}   \\
\trace(\mA^T\mB)      &= \mathbf{1}^T(\mA\circ\mB)\mathbf{1}                          \\
                      &= \sum_{i,j} \mA_{ij}\mB_{ij}
\end{align}


\chapter{Derivatives}

\section{Useful Derivatives}
For general $\mA$ and $\mX$ (no special structure):
\begin{align}
\partial\mA           &= 0~~\textrm{where $\mA$ is a constant} \\
\partial(c\mX)        &= c\partial\mX                          \\
\partial(\mX+\mY)     &= \partial\mX+\partial\mY               \\
\partial(\trace(\mX)) &= \trace(\partial(\mX))                 \\
\partial(\mX\mY)      &= (\partial\mX)\mY + \mX(\partial\mY)   \\
\partial(\mX\circ\mY) &= (\partial\mX)\circ\mY + \mX\circ(\partial\mY) \\
%TODO Kronecker x in circle equation 39 Matrix Cookbook
\partial(\mX^{-1})    &= -\mX^{-1}(\partial\mX)\mX^{-1}        \\
\partial(\det(\mX))   &= \trace(\textrm{adj}(\mX)\partial\mX)  \\
\partial(\det(\mX))   &= \det(\mX)\trace(\mX^{-1}\partial\mX)  \\
\partial(\ln(\det(\mX))) &= \trace(\mX^{-1}\partial\mX)        \\
\partial(\mX^T)       &= (\partial\mX)^T                       \\
\partial(\mX^H)       &= (\partial\mX)^H                       
\end{align}

\chapter{Matrix Rogue Gallery}

\section{Non-Singular vs.\ Singular Matrices}
For $\mA\in\sRnn$ (initially drawn from \citep[p.\ 574]{Strang2016}):
\begin{center}
\begin{tabular}{ll}
\textbf{Non-Singular}                           & \textbf{Singular}                        \\
$\mA$ is invertible                             & $\mA$ is not invertible                  \\
The columns are independent                     & The columns are dependent                \\
The rows are independent                        & The rows are dependent                   \\
$\det(\mA)\ne0$                                 & $\det(\mA)=0$                            \\
$\mA\vx=0$ has one solution: $\vx=0$            & $\mA\vx=0$ has infinitely many solutions \\
$\mA\vx=\vb$ has one solution: $\vx=\mA^{-1}\vb$& $\mA\vx=\vb$ has either no or infinitely many solutions \\
$\mA$ has $n$ nonzero pivots                    & $\mA$ has $r<n$ pivots                   \\
$\mA$ has full rank $r=n$                       & $\mA$ has rank $r<n$                     \\
The reduced row echelon form is $\mR=\mI$       & $\mR$ has at least one zero row          \\
The column space is all of $\sRn$               & The column space has dimension $r<n$     \\
The row space is all of $\sRn$                  & The row space has dimension $r<n$        \\
All eigenvalues are nonzero                     & Zero is an eigenvalue of $\mA$           \\
$\mA^T\mA$ is symmetric positive definite       & $\mA^T\mA$ is only semidefinite          \\
$\mA$ has $n$ positive singular values          & $\mA$ has $r<n$ singular values        
\end{tabular}
\end{center}

\section{Diagonal Matrix}

\begin{center}
\includegraphics[width=1.5in]{imgs/rg_diagonal.pdf}
\end{center}

\begin{equation}
A=\diag(a_1,\ldots,a_n)=
\begin{bmatrix}
a_1  &        &  \\
     & \ddots &  \\
     &        & a_n 
\end{bmatrix}
\end{equation}

Square matrix. Entries above diagonal are equal to entries below diagonal.

Number of ``free entries": $\frac{n(n+1)}{2}$.

\subsection*{Special Properties}

\begin{align}
\eig(A) &= {a_1,\ldots,a_n}           \\
\det(A) &= \prod_i a_i                \\
A^{-1}  &= 
\begin{bmatrix}
\frac{1}{a_1} &        &               \\
              & \ddots &               \\
              &        & \frac{1}{a_n}
\end{bmatrix} \\
\vx^T \mA \vx &= \sum_i a_i \vx_i^2     \\
\end{align}




\section{Dyads}

\begin{center}
\includegraphics[width=2in]{imgs/rg_dyad.pdf}
\end{center}

$\mA\in\sRmn$ is a dyad if it can be written as
\begin{equation}
\mA=\vu\vv^T~~~\vu\in\sRm, \vv\in\sRn
\end{equation}

\subsection*{Special Properties}
\begin{itemize}
\item The columns of $\mA$ are copies of $\vu$ scaled by the values of $\vv$.
\item The rows of $\mA$ are copies of $\vu^T$ scaled by the values of $\vv$.
\item If $\mA$ is a dyad, it acts on a vector $\vx$ as $\mA\vx=(\vu\vv^T)\vx=(\vv^T\vu)\vx$.
\item $\mA\vx=c\vu$ ($\mA$ scales $\vx$ and points it along $\vu$).
\item $\mA_{ij}=\vu_i\vv_j$.
\item If $\vu,\vv\ne0$, then $\rank(\mA)=1$.
\item If $m=n$, $\mA$ has one eigenvalue $\lambda=\vv^T\vu$ and eigenvector $\vu$.
\item A dyad can always be written in a normalized form $c\tilde\vu\tilde\vv^T$.
\end{itemize}
%TODO: Dyad eigenvalues



\section{Hermitian Matrix}
$\mH\in\sCmn$ is Hermitian iff
\begin{equation}
\mH=\mH^H
\end{equation}
where $\mH^H$ is the conjugate transpose of $\mH$.

For $\mH\in\sRmn$, Hermitian and symmetric matrices are equivalent.

\subsection*{Special Properties}
\begin{align}
\mH_{ii} &\in \sR      \\
\mH\mH^H &=   \mH^H\mH \\
\vx^H\mH\vx &\in \sR~~\forall\vx\in\sC \\
\mH_1+\mH_2 &= \textrm{Hermitian} \\
\mH^{-1}    &= \textrm{Hermitian} \\
\mA+\mA^H   &= \textrm{Hermitian} \\
\mA-\mA^H   &= \textrm{Skew-Hermitian} \\
\mA\mB      &= \textrm{Hermitian iff $\mA\mB=\mB\mA$} \\
\det(\mH)   &\in \sR \\
\eig(\mH)   &\in \sR
\end{align}



\section{Idempotent Matrix}
A matrix $\mA$ is idempotent iff
\begin{equation}
\mA\mA=\mA
\end{equation}

\subsection*{Special Properties}
\begin{align}
\mA^n        &=A~~\forall n              \\
\mI-\mA      &~~\textrm{is idempotent}   \\
\mA^H        &~~\textrm{is idempotent}   \\
\mI-\mA^H    &~~\textrm{is idempotent}   \\
\rank(\mA)   &= \trace(\mA)              \\
\mA(I-\mA)   &= 0                        \\
\mA\pinv     &= \mA                      \\
f(s\mI+t\mA) &= (\mI-\mA)f(s)+\mA f(s+t) \\
\mA\mB=\mB\mA&\implies \mA\mB~\textrm{is idempotent} \\
\eig(\mA)_i  &\in \{0,1\} \\
\mA & \textrm{~is always diagonalizable}
\end{align}
$\mA-\mI$ may not be idempotent.



\section{Laplacian Matrix of a Graph}
Let $\mL$ be the Laplacian matrix of a graph $G$ with neither multiple edges nor loops defined by $(V,E,w)$ where $V$ is the set of vertices, $E$ the set of edges, and $w$ is a weight function. Is is also the case that $L=D-A$ where $D$ is the degree matrix and $A$ is the adjaceny matrix. In the case of directed graphs either the indegree or outdegree might be used.

The elements of $\mL$ are given by
\begin{equation}
\mL_{i,j} = \begin{cases}
  \deg(v_i) & \textrm{if $i=j$} \\
  -1        & \textrm{if $i\ne j$ and $v_i$ is adjacent to $v_j$} \\
   0        & \textrm{otherwise}
\end{cases}
\end{equation}

If $G$ is weighted, the elements of its Laplacian $\mL$ are given by
\begin{equation}
\mL_{i,j} = \begin{cases}
  \sum_{j,j\ne i} w(i,j) & \textrm{if $i=j$} \\
  -w(i,j)   & \textrm{if $i\ne j$ and $v_i$ is adjacent to $v_j$ with weight $w(i,j)$} \\
   0        & \textrm{otherwise}
\end{cases}
\end{equation}

For an undirected graph $G$ and its Laplacian $\mL$:
\begin{itemize}
\item $\mL$ is symmetric
\item $L\ispsd 0$
\item The row sum and column sums of $\mL$ are both zero.
\item $\mL$ is singular
\item The number of connected components in $G$ is the dimension of $\ns(L)$ and the algebraic multiplicity of the 0 eigenvalue.
\item The smallest non-zero eigenvalue of $\mL$ is called the spectral gap.
\item The second smallest eigenvalue of $\mL$ (could be zero) is the algebraic connectivity (Fiedler value) of $G$ and approximates the sparest cut of $G$.
\item For a graph with multiple connected components, $\mL$ is a block diagonal matrix.
\item Using preconditioners, the linear equaitons of any Laplacian matrix $\mL\in\sRnn$ can be solved to accuracy $\epsilon$ in time $O((\nnz(\mL) + n\log n (\log \log n)^2) \log \epsilon^{-1})$. The best balance between preconditioners and solving linear equations yields an algorithm of complexity $O(\nnz(\mL) \log^c n \log \epsilon^{-1})$.~\citep{Spielman2010}
\end{itemize}

\begin{align}
\vx^T \mL \vx = \sum_{(u,v)\in E} w(u,v) \left(\vx(u)-\vx(v)\right)^2~~~~\vx\in\sR^{V} \label{equ:laplace_quad}
\end{align}

\autoref{equ:laplace_quad} provides a measure of the ``smoothness" of $\vx$ over the edges of $G$. The more $\vx$ jumps over an edge, the larger the quadratic form becomes.





\section{Orthogonal Matrix}

\begin{center}
\includegraphics[width=1.5in]{imgs/rg_orthogonal.pdf}

(Not much visible structure)
\end{center}


\begin{equation}
U=
\begin{bmatrix}
1 & 0 & 0 & 0 & 0 & 0 \\
0 & 0 & 0 & 0 & 1 & 0 \\
0 & 0 & 1 & 0 & 0 & 0 \\
0 & 1 & 0 & 0 & 0 & 0 \\
0 & 0 & 0 & 0 & 0 & 1 \\
0 & 0 & 0 & 1 & 0 & 0 \\
\end{bmatrix}
\end{equation}

A matrix $\mU$ is orthogonal iff:

\begin{equation}
\mU^T \mU = \mU \mU^T = I
\end{equation}

Square matrix. The columns form an orthonormal basis of $\mathbb{R}^n$.



\subsection*{Special Properties}

\begin{itemize}
\item The eigenvalues of $\mU$ are placed on the unit circle.
\item The eigenvectors of $\mU$ are unitary (have length one).
\item $\mU^{-1}$ is orthogonal.
\item The product of two orthogonal matrices is itself orthogonal.
\end{itemize}

\begin{align}
\mU^T     &= \mU^{-1} \\
\mU^{-T}  &= \mU      \\
\mU^T\mU  &= \mI      \\
\mU\mU^T  &= \mI      \\
\det(\mU) &= \pm1
\end{align}



Orthogonal matrices preserve the lengths and angles of the vectors they operator on. The converse is true: any matrix which preserves lengths and angles is orthogonal.
\begin{equation}
\norm{\mU \vx}^2_2=(\mU\vx)^T(\mU\vx)=\vx^T\mU^T\mU\vx=\vx^T\vx=\norm{\vx}^2_2~~\forall \vx
\end{equation}
\begin{equation}
\norm{\mU \mA \mV}_F=\norm{\mA}_F~~\forall \mA,\mU,\mV~\textrm{with}~\mU,\mV \textrm{orthogonal}
\end{equation}



\section{Permutation Matrix}
\begin{center}
\includegraphics[width=1.5in]{imgs/rg_permutation_matrix.pdf}
\end{center}

TODO



\section{Positive Definite}

$\mP\in\sSn$ is positive definite (denoted $\mP\ispd0$) if any of the following are true:
\begin{itemize}
\item $\vx^T\mP\vx>0,\forall\vx\in\sRn$.
\item $\eig(\mP)>0$
\end{itemize}


\subsection*{Special Properties}

\begin{itemize}
\item $\mP^{-1}\ispd0$
\item $c\mP\ispd0$
\item $\mA_{ii}\in\sR$
\item $\mA_{ii}>0$
\item $\trace(\mP)\ge0$. %TODO: Shouldn't this be >0?
\item $\det(\mP)>0$
\item The eigenvalues of $\mP^{-1}$ are the inverses of the eigenvalues of $\mP$.
\item For $\mP\in\sRmn$, $\mP^T\mP\ispd0\iff \mP$ is full-column rank ($\rank(\mP)=n$)
\item For $\mP\in\sRmn$, $\mP\mP^T\ispd0\iff \mP$ is full-row rank ($\rank(\mP)=m$)
\end{itemize}

\subsubsection{Ellipsoids}
$\mP\ispd0$ defines a full-dimensional, bounded ellipsoid defined by the set
\begin{equation}
\mathcal{E}=\{\vx\in\sRn: (\vx-\vz)^T\mP^{-1}(\vx-\vz)\le \beta\}
\end{equation}
The eigenvectors of $\mP$ define the directions of the semi-axes of the ellipsoid; the lengths of these axes are given by $\sqrt{\beta\lambda_i}$ where $\lambda_i$ are the eigenvalues of $\mP$. The ellipsoid is centered at $\vz$. Since $\mP\ispd 0 \implies \mP^{-1}\ispd 0$, the Cholesky decomposition says that $\mP^{-1}=\mA^T\mA$; therefore, an equivalent definition of the ellipsoid is $\mathcal{E}=\{\vx\in\sRn: \norm{\mA\vx}_2\le1\}$.

\section{Positive Semi-Definite}

$\mA$ is positive semi-definite (denoted $\mA\ispsd0$) if any of the following are true:
\begin{itemize}
\item $\vx^T\mA\vx\ge0,\forall\vx\in\sRn$.
\item $\eig(\mA)\ge0$
\end{itemize}

\subsection*{Special Properties}
\begin{itemize}
\item For $\mA\in\sRmn$, $\mA^T\mA\ispsd0$
\item For $\mA\in\sRmn$, $\mA\mA^T\ispsd0$
\item The positive semi-definite matrices $\sPSD$ form a convex cone. For any two PSD matrices $\mA,\mB\in\sPSD$ and some $\alpha\in[0,1]$:
\begin{equation}
\vx^T(\alpha\mA+(1-\alpha)\mB)\vx=\alpha \vx^T\mA\vx+(1-\alpha)\vx^T\mB\vx\ge0~~\forall\vx
\end{equation}
\begin{equation}
\alpha\mA+(1-\alpha)\mB\in\sPSD
\end{equation}
\item For $\mA\in\sPSD$ and $\alpha\ge0$, $\alpha\mA\ispsd0$, so $\sPSD$ is a cone.
\item $\mA\ispsd 0$ has a unique PSD matrix $\mS^{1/2}$ such that $\mS^{1/2}\mS^{1/2}=\mA$
\end{itemize}

\subsection{Loewner order}
If $\mA-\mB\ispsd 0$, then we say $\mA\ispsd \mB$. A sufficient condition for this is that $\lambda_n(\mA)\ge\lambda_1(\mB)$.



\section{Projection Matrix}
A square matrix $\mP$ is a projection matrix that projects onto a vector space $\mathcal{S}$ iff
\begin{align}
\mP&~\textrm{is idempotent} \\
\mP\vx&\in\mathcal{S}~~\forall\vx \\
\mP\vz&=\vz~~\forall\vz\in\mathcal{S}
\end{align}


\section{Single-Entry Matrix}
\label{sec:rogue_single_entry} 
\begin{equation}
\mJ^{2,3} =
\begin{bmatrix}
0 & 0 & 0 & 0 \\
0 & 0 & 1 & 0 \\
0 & 0 & 0 & 0 \\
0 & 0 & 0 & 0
\end{bmatrix}
\end{equation}

The single-entry matrix $\mJ^{iJ}\in\sRnn$ is defined as the matrix which is zero everywhere except for the entry $(i,j)$, which is $1$.


%TODO: Much material from MCB



\section{Singular Matrix}
A square matrix that is not invertible.

$\mA\in\sRnn$ is singular iff $\det \mA=0$ iff $\mathcal{N}(A)\ne\{0\}$.


\section{Symmetric Matrix}

\begin{center}
\includegraphics[width=1.5in]{imgs/rg_symmetric_matrix.pdf}
\end{center}

$\mA\in\sSn$ is a symmetric matrix if $\mA=\mA^T$ (entries above diagonal are equal to entries below diagonal).

\begin{equation}
\begin{bmatrix}
a & b & c & d & e & f \\
b & g & l & m & o & p \\
c & l & h & n & q & r \\
d & m & n & i & s & t \\
e & o & q & s & j & u \\
f & p & r & t & u & k \\
\end{bmatrix}
\end{equation}


\subsection*{Special Properties}

\begin{align}
\mA                                &=   \mA^T \\
\eig(A)                            &\in \sRn  \\
\textrm{Number of ``free entries"} &=    \frac{n(n+1)}{2}
\end{align}

If $\mA$ is real, it can be decomposed into $\mA=\mQ^T\mD\mQ$ where $\mQ$ is a real orthogonal matrix (the columns of which are eigenvectors of $\mA$) and $\mD$ is real and diagonal containing the eigenvalues of $\mA$.

For a real, symmetric matrix with non-negative eignevalues, the eigenvalues and singular values coincide.



\section{Skew-Hermitian}
A matrix $\mH\in\sCmn$ is Skew-Hermitian iff
\begin{equation}
\mH=-\mH^H
\end{equation}



\section{Toeplitz Matrix, General Form}

\begin{center}
\includegraphics[width=1.5in]{imgs/rg_toeplitz.pdf}
\end{center}
Constant values on descending diagonals.
\begin{equation}
\begin{bmatrix}
  a_{0} & a_{-1} & a_{-2} & \ldots  & \ldots & a_{-(n-1)}  \\
  a_{1} & a_0    & a_{-1} & \ddots  &        & \vdots \\
  a_{2} & a_{1}  & \ddots & \ddots  & \ddots & \vdots \\ 
 \vdots & \ddots & \ddots & \ddots  & a_{-1} & a_{-2}\\
 \vdots &        & \ddots & a_{1}   & a_{0}  & a_{-1} \\
a_{n-1} & \ldots & \ldots & a_{2}   & a_{1}  & a_{0}
\end{bmatrix}
\end{equation}


\section{Toeplitz Matrix, Discrete Convolution}

\begin{center}
\includegraphics[width=1.5in]{imgs/rg_toeplitz_1d_conv.pdf}
\end{center}

Constant values on main and subdiagonals.

\begin{equation}
\begin{bmatrix}
  h_m &   0 &   0 &      \hdots &   0 &   0 \\
  \vdots & h_m &   0 &   \hdots &   0 &   0 \\
  h_1 & \vdots & h_m &   \hdots &   0 &   0 \\
    0 & h_1 & \ddots & \ddots &   0 &   0 \\
    0 &   0 & h_1 &    \ddots & h_m &   0 \\
    0 &   0 &   0 &    \ddots & \vdots & h_m \\
    0 &   0 &   0 &      \hdots & h_1 & \vdots \\
    0 &   0 &   0 &      \hdots &   0 & h_1 
\end{bmatrix}
\end{equation}


\section{Triangular Matrix}

\begin{center}
\includegraphics[width=1.5in]{imgs/rg_lower_triangular.pdf}~\includegraphics[width=1.5in]{imgs/rg_upper_triangular.pdf}
\end{center}

\begin{equation}
\begin{bmatrix}
a & b & c & d & e & f \\
  & g & h & i & j & k \\
  &   & l & m & n & o \\
  &   &   & p & q & r \\
  &   &   &   & s & t \\
  &   &   &   &   & u \\
\end{bmatrix}
~
~
\begin{bmatrix}
a &   &   &   &   &   \\
b & g &   &   &   &   \\
c & h & l &   &   &   \\
d & i & m & p &   &   \\
e & j & n & q & s &   \\
f & k & o & r & t & u \\
\end{bmatrix}
\end{equation}

Square matrices in which all elements either above or below the main diagonal are zero. An upper (left) and a lower (right) triangular matrix are shown above.

For an upper triangular matrix $A_{ij}=0$ whenever $i>j$; for a lower triangular matrix $A_{ij}=0$ whenever $i<j$.


\subsection*{Special Properties}

\begin{align}
\eig(A) &= \diag(A)             \\
\det(A) &= \prod_i \diag(A)_i
\end{align}

The product of two upper (lower) triangular matrices is still upper (lower) triangular.

The inverse of a nonsingular upper (lower) triangular matrix is still upper (lower) triangular.


\section{Tridiagonal Matrix}

\begin{center}
\includegraphics[width=1.5in]{imgs/rg_tridiagonal.pdf}
\end{center}

\begin{equation}
\begin{bmatrix}
b_1 & c_1 &     &        &        &         \\
a_2 & b_2 & c_2 &        &        &         \\
    & a_3 & b_3 & c_3    &        &         \\
    &     & a_4 & b_4    & \ddots &         \\
    &     &     & \ddots & \ddots & c_{n-1} \\
    &     &     &        & a_n    & b_n     \\
\end{bmatrix}
\end{equation}

A tridiagonal matrix has values on its main diagonal as well as the diagonals abutting the main, with zeros elsewhere.

A system of $n$ unknowns which can be written as
\begin{align}
a_i x_{i-1}+b_i x_i + c_i x_{i+1} &= d_i \\
a_1 &=0                                  \\
c_n &=0
\end{align}
can be rewritten as
\begin{equation}
\begin{bmatrix}
b_1 & c_1 &     &        &        &         \\
a_2 & b_2 & c_2 &        &        &         \\
    & a_3 & b_3 & c_3    &        &         \\
    &     & a_4 & b_4    & \ddots &         \\
    &     &     & \ddots & \ddots & c_{n-1} \\
    &     &     &        & a_n    & b_n     \\
\end{bmatrix}
\begin{bmatrix}
x_1 \\ x_2 \\ x_3 \\ \vdots \\ x_n
\end{bmatrix}
=
\begin{bmatrix}
d_1 \\ d_2 \\ d_3 \\ \vdots \\ d_n
\end{bmatrix}
\end{equation}
This system can be solved in $O(n)$ time using the tridiagonal matrix algorithm (aka the Thomas Algorithm). The algorithm is not unconditionally stable; however, it is stable when the matrix is diagonally dominant or symmetric positive definite. A matix is diagonally dominant if for every row of the matrix the agnitude of the diagoanl entry is greater than or equal to the sum of the magnitudes of all the other non-diagonal entries in that row ($|a_{ii}|\ge\sum_{j\ne i} |a_{ij}|~\forall i$). If uncondonitional stability is grequired, Gaussian elimination with partial pivoting is an alternative, if slower, solution method. See \citep[Theorem 9.12]{Higham2002} for full stability details.

A modified system can be solved for situations involving periodic boundary conditions, e.g.:
\begin{align}
a_1 x_n + b_1 x_1 + c_1 x_2 &= d_1 \\
a_i x_{i-1} + b_i x_i + c_i x_{i+1} &= d_i~~\forall i=2,\ldots,n-1 \\
a_n x_{n-1}+b_n x_n + c_n x_1 &= d_n 
\end{align}

Modified algorithms are also available for block tridiagonal matrices~\citep[\textsection3.8]{Quateroni2007}. See \citep[\textsection5.5]{Gallopoulos2016} for a discussion of parallel solvers.


\section{Vandermonde Matrix}
\begin{equation}
V=
\begin{bmatrix}
1      & \alpha_1 & \alpha_1^2 & \dots  & \alpha_1^{n-1} \\
1      & \alpha_2 & \alpha_2^2 & \dots  & \alpha_2^{n-1} \\
1      & \alpha_3 & \alpha_3^2 & \dots  & \alpha_3^{n-1} \\
\vdots & \vdots   & \vdots     & \ddots & \vdots         \\
1      & \alpha_m & \alpha_m^2 & \dots  & \alpha_m^{n-1}
\end{bmatrix}
\end{equation}
Alternatively,
\begin{equation}
V_{i,j} = \alpha_i^{j-1}
\end{equation}

\subsection*{Uses}
Polynomial interpolation of data.

\subsection*{Special Properties}
\begin{itemize}
\item $\det(V)=\prod_{1\le i < j \le n} (x_j-x_i)$
\end{itemize}

\chapter{Matrix Decompositions}

\section{LLT/UTU: Cholesky Decomposition} %TODO: Add UU to TOC

\begin{center}
\includegraphics[align=c,height=1in]{imgs/decomp_cholesky_a.pdf}\textbf{\large =}
\includegraphics[align=c,height=1in]{imgs/decomp_cholesky_L.pdf}\textbf{\large *}
\includegraphics[align=c,height=1in]{imgs/decomp_cholesky_LT.pdf}
\end{center}

If $\mA$ is symmetric, positive definite, square, then
\begin{equation}
\mA=\mU^T\mU=\mL\mL^T
\end{equation}
where $\mU$ is a unique upper triangular matrix and $\mL$ is a unique lower-triangular matrix.

\section{LDL Decomposition}

\begin{center}
\includegraphics[align=c,height=1in]{imgs/decomp_ldlt_a.pdf}\textbf{\large =}
\includegraphics[align=c,height=1in]{imgs/decomp_ldlt_L.pdf}\textbf{\large *}
\includegraphics[align=c,height=1in]{imgs/decomp_ldlt_D.pdf}\textbf{\large *}
\includegraphics[align=c,height=1in]{imgs/decomp_ldlt_LT.pdf}
\end{center}

This is a special case of the LDM decomposition.\footnote{TODO: Crossreference}
If $\mA$ is a non-singular symmetric definite square matrix, then
\begin{equation}
\mA=\mL\mD\mL^T=\mL^T\mD\mL
\end{equation}
where $\mL$ is a unit lower triangular matrix and $\mD$ is a diagonal matrix. If $\mA\ispd0$, then $\mD_{ii}>0$.

% For $\mA\in\sRnn$,
% \begin{align}
% \mA&\ispsd0\iff \exists\mB\ispsd0: \mA=\mB^2 \\
% \mA&\ispd0 \iff \exists\mB\ispd 0: \mA=\mB^2
% \end{align}
% where $\mB$ is called the ``matrix square-root" of $\mA$.

% For $\mA\ispsd0$, we can use the spectral factorization $\mA=\mU\mD\mU^T$ and take $\mD^{1/2}=\diag(\sqrt{\lambda_1},\ldots,\sqrt{\lambda_n})$ to get $\mB=\mU\mD^{1/2}\mU^T$.


\section{PCA: Principle Components Analysis}
Find normalized directions in data space such that the variance of the projections of the centered data points is maximal. For centered data $\tilde \mX$, the mean-square variation of data along a vector $\vx$ is $\vx^T \tilde \mX \tilde \mX^T \vx$.
\begin{equation}
\max_{\vx\in\sRn,~\norm{\vx}_2=1} \vx^T \tilde \mX \tilde \mX^T \vx
\end{equation}
Taking an SVD of $\tilde \mX \tilde \mX^T$ gives $H=\mU_r\mD^2\mU^T$, which is maximized by taking $\vx=\vu_1$. By repeatedly removing the first principal components and recalculating, all the principal axes can be found.




\section{QR: Orthogonal-triangular}

\begin{center}
\includegraphics[align=c,height=1in]{imgs/decomp_qr_a.pdf}\textbf{\large =}
\includegraphics[align=c,height=1in]{imgs/decomp_qr_q.pdf}\textbf{\large *}
\includegraphics[align=c,height=1in]{imgs/decomp_qr_r.pdf}
\end{center}

For $\mA\in\sRnn$, $\mA=\mQ\mR$ where $\mQ$ is orthogonal and $\mR$ is an upper triangular matrix. If $\mA$ is non-singular, then $\mQ$ and $\mR$ are uniquely defined if $\diag(\mR)$ are imposed to be positive.

\subsection*{Algorithms}

Gram-Schmidt.




\section{SVD: Singular Value Decomposition}

\begin{center}
\includegraphics[align=c,height=1in]{imgs/decomp_svd_a.pdf}\textbf{\large =}
\includegraphics[align=c,height=1in]{imgs/decomp_svd_u.pdf}\textbf{\large *}
\includegraphics[align=c,height=1in]{imgs/decomp_svd_s.pdf}\textbf{\large *}
\includegraphics[align=c,height=1in]{imgs/decomp_svd_v.pdf}
\end{center}

\begin{center}
\includegraphics[align=c,width=0.5in]{imgs/decomp_svd_a_compact.pdf}\textbf{\large =}
\includegraphics[align=c,width=0.5in]{imgs/decomp_svd_u_compact.pdf}\textbf{\large *}
\includegraphics[align=c,width=0.5in]{imgs/decomp_svd_s_compact.pdf}\textbf{\large *}
\includegraphics[align=c,width=0.5in]{imgs/decomp_svd_v_compact.pdf}
\end{center}

Any matrix $\mA\in\sRmn$ can be written as
\begin{equation}
\mA=\mU \mD \mV^T=\sum_{i=1}^r \sigma_i u_i v_i^T
\end{equation}
where
\begin{align}
\mU&=\textrm{eigenvectors of~}\mA\mA^T & \sRmm \\
\mD&=\diag(\sigma_i)=\sqrt{\diag(\eig(\mA\mA^T))}      & \sRmn \\
\mV&=\textrm{eigenvectors of~}\mA^T\mA & \sRnn
\end{align}
Let $\sigma_i$ be the non-zero singular values for $i=1,\ldots,r$ where $r$ is the rank of $\mA$; $\sigma_1\ge\ldots\ge\sigma_r$.

We also have that
\begin{align}
\mA   \vv_i &= \sigma_i \vu_i \\
\mA^T \vu_i &= \sigma_i \vv_i \\
\mU^T\mU &= \mI \\
\mV^T\mV &= \mI
\end{align}

$\mD$ can be written in an expanded form:
\begin{equation}
\tilde \mD=
\begin{bmatrix}
\mD       & 0_{r,n-r}   \\
0_{m-r,r} & 0_{m-r,n-r}
\end{bmatrix}
\end{equation}
The final $n-r$ columns of $\mV$ give an orthonormal basis spanning $\ns(\mA)$. An orthonormal basis spanning the range of $\mA$ is given by the first $r$ columns of $\mU$.

\begin{align}
\norm{\mA}^2_F&=\textrm{Frobenius norm} =\trace(\mA^T\mA)=\sum_{i=1}^r \sigma_i^2 \\
\norm{\mA}^2_2&=\sigma_1^2 \\
\norm{\mA}_* &= \textrm{nuclear norm}=\sum_{i=1}^r \sigma_i
\end{align}

The \textbf{condition number} $\kappa$ of an invertible matrix $\mA\in\sRnn$ is the ratio of the largest and smallest singular value. Matrices with large condition numbers are closer to being singular and more sensitive to changes.
\begin{equation}
\kappa(\mA)=\frac{\sigma_1}{\sigma_n}=\norm{A}_2 \cdot \norm{A^{-1}}_2
\end{equation}

\subsection*{Low-Rank Approximation}
Approximating $\mA\in\sRmn$ by a matrix $\mA_k$ of rank $k>0$ can be formulated as the optimization probem
\begin{equation}
\min_{\mA_k\in\sRmn} \norm{\mA-\mA_k}_F^2: \rank{\mA_k}=k, 1\le k \le \rank(\mA)
\end{equation}
The optimal solution of this problem is given by
\begin{equation}
\mA_k=\sum_{i=1}^k \sigma_i \vu_i \vv_i^T
\end{equation}
where
\begin{align}
\frac{\norm{\mA_k}_F^2}{\norm{\mA}_F^2}&=\frac{\sigma_1^2+\ldots+\sigma_k^2}{\sigma_1^2+\ldots+\sigma_r^2} \\
1-\frac{\norm{\mA_k}_F^2}{\norm{\mA}_F^2}&=\frac{\sigma_{k+1}^2+\ldots+\sigma_r^2}{\sigma_1^2+\ldots+\sigma_r^2}
\end{align}
is the fraction of the total variance in $\mA$ explained by the approximation $\mA_k$.

\subsection*{Range and Nullspace}
\begin{align}
\ns(\mA) &= \range(\mV_{nr})                      \\
\ns(\mA)^\perp \equiv\range(\mA^T)&=\range(\mV_r) \\
\range(\mA)&=\range(\mU_r)                        \\
\range(\mA)^\perp\equiv\ns(\mA^T)&=\range(\mU_{nr})
\end{align}
where $\mV_r$ is the first $r$ columns of $V$ and $V_nr$ are the last $[r+1,n]$ columns; similarly for $\mU$.


\subsection*{Projectors}
The projection of $\vx$ onto $\ns(\mA)$ is $(\mV_{nr}\mV_{nr}^T)\vx$. Since $\mI_n=\mV_r\mV_r^T+\mV_{nr}\mV_{nr}^T$, $(\mI_n-\mV_{r}\mV_{r}^T)\vx$ also works. The projection of $\vx$ onto $\range(\mA)$ is $(\mU_r\mU_r^T)\vx$.

If $\mA\in\sRmn$ is full row rank ($\mA\mA^T\ispd0$), then the minimum distance to an affine set $\{x:\mA\vx=\vb\}, \vb\in\sRm$ is given by $\vx^*=\mA^T(\mA\mA^T)^{-1}\vb$. %TODO

If $\mA\in\sRmn$ is full column rank ($\mA^T\mA\ispd0$), then the minimum distance to an affine set $\{x:\mA\vx=\vb\}, \vb\in\sRm$ is given by $\vx^*=\mA(\mA^T\mA)^{-1}\mA^T\vb$. %TODO


\subsection*{Computational Notes}
A \textit{numerical rank} can be estimated for the matrix as the largest $k$ such that $\sigma_k>\epsilon \sigma_1$ for $\epsilon\ge0$.



\section{Eigenvalue Decomposition for Diagonalizable Matrices}

For a square, diagonalizable matrix $\mA\in\mathbb{R}^{n,n}$
\begin{equation}
\mA=U\Lambda U^{-1}
\end{equation}
where $U\in\mathbb{C}^{n,n}$ is an invertible matrix whose columns are the eigenvectors of $\mA$ and $\Lambda$ is a diagonal matrix containing the eigenvalues $\lambda_1,\ldots,\lambda_n$ of $\mA$ in the diagonal.

The columns $\vu_1,\ldots,\vu_n$ satisfy
\begin{equation}
\mA \vu_i=\lambda_i \vu_i~~i=1,\ldots,n
\end{equation}

\section{Eigenvalue (Spectral) Decomposition for Symmetric Matrices}

A symmetric matrix $\mA\in\mathbb{R}^{n,n}$ can be factored as
\begin{equation}
\mA=U\Lambda U^T=\sum_i^n \lambda_i \vu_i \vu_i^T
\end{equation}
where $U\in\mathbb{R}^{n,n}$ is an orthogonal matrix whose columns $\vu_i$ are the eigenvectors of $\mA$ and $\Lambda$ is a diagonal matrix containing the eigenvalues $\lambda_1\ge\ldots\ge\lambda_n$ of $\mA$ in the diagonal. These eigenvalues are always real. The eigenvectors can always be chosen to be real and to form an orthonormal basis.

The columns $\vu_1,\ldots,\vu_n$ satisfy
\begin{equation}
\mA \vu_i=\lambda_i \vu_i~~i=1,\ldots,n
\end{equation}


\section{Schur Complements}

For $\mA\in\sSn$, $\mB\in\sSn$, $\mX\in\sRnm$ with $\mB\ispd0$ and the block matrix
\begin{equation}
\mM=
\begin{bmatrix}
\mA & \mX \\
\mX^T & \mB
\end{bmatrix}
\end{equation}
and the Schur complement of $\mA$ in $\mM$
\begin{equation}
S=\mA-\mX\mB^{-1}\mX^T
\end{equation}
Then
\begin{align}
\mM\ispsd0&\iff S\ispsd0 \\
\mM\ispd0 &\iff S\ispd0
\end{align}


\chapter{Eigenvalue Properties}

$\lambda\in\mathbb{C}$ is an eigenvalue of $\mA\in\sRnn$ and $u\in\mathbb{C}^n$ is a corresponding eigenvector if $\mA\vu=\lambda\vu$ and $\vu\ne0$. Equivalantly, $(\lambda \mI_n-\mA)\vu=0$ and $\vu\ne0$. Eigenvalues satisfy the equation $\det(\lambda\mI_n-\mA)=0$.

Any matrix $\mA\in\sRnn$ has $n$ eigenvalues, though some may be repeated. $\lambda_1$ is the largest eigenvalue and $\lambda_n$ the smallest.

If $\lambda$ is an eigenvalue of $\mA$, $\lambda^2$ is an eigenvalue of $\mA^2$.

\begin{equation}
\eig(\mA\mA^T)=\eig(\mA^T\mA)
\end{equation}
(Note that the number of entries in $\mA\mA^T$ and $\mA^T\mA$ may differ significantly leading to different compute times.)

\begin{equation}
\eig(\mA^T\mA)\ge0
\end{equation}

\begin{equation}
\lambda_\textrm{min}(\mA)\le \frac{\vx^T \mA \vx}{\vx^T\vx} \le \lambda_\textrm{max}(\mA)~~\vx\ne0
\end{equation}

\section{Weyl's Inequality}
If $\mM,\mH,\mP\in\sRnn$ are Hermitian matrices and $\mM=\mH+\mP$ ($\mH$ is perturbed by $\mP$) and $\mM$ has eigenvalues $\mu_1\ge\cdots\ge\mu_n$, $\mH$ has eigenvalues $\nu_1\ge\cdots\ge\nu_n$, and $\mP$ has eigenvalues $\rho_1\ge\cdots\ge\rho_n$, then
\begin{equation}
\nu_i+\rho_n\le \mu_i \le \nu_i + \rho_1~\forall i
\end{equation}
If $j+k-n\ge i \ge r+s-1$, then
\begin{equation}
\nu_j+\rho_k\le\mu_i\le\nu_r+\rho_s
\end{equation}
If $\mP\ispsd0$, then $\mu_i>\nu_i~\forall i$.

%TODO
% \section*{Computation}
% TODO: eigsh, small eigen value extraction, top-k

\section{Estimating Eigenvalues}
\subsection{Gershgorin circle theorem}
For $\mA\in\sCnn$ with entries $a_{ij}$ let $R_i=\sum_{j\ne i} |a_{ij}|$ be the sum of the absolute values of the non-diagonal entries of the $i$-th row. Let $D(a_{ii},R_i)\subseteq\sC$ be a closed disc (a circle containing its boundary) centered at $a_{ii}$ with radius $R_i$. This is the Gershgorin disc.

Every eigenvalue of $\mA$ lies within at least one of the $D(a_{ii},R_i)$. Further, if the union of $k$ such discs is disjoint from the union of the other $n-k$ discs then the former union contains exactly $k$ and the latter $n-k$ of the eigenvalues of $\mA$.

\chapter{Norms}

\section{Matrices}
Matrix norms satisfy some properties:
\begin{align}
f(\mA)    &\ge 0             \\
f(\mA)    &=   0  \iff \mA=0 \\
f(c\mA)   &=   |c|f(\mA)     \\
f(\mA+\mB)&\le f(\mA)+f(\mB)
\end{align}
Many popular matrix norms also satisfy ``sub-multiplicativity": $f(\mA\mB)\le f(\mA)f(\mB)$.

\subsection{Frobenius norm}
\begin{align}
\norm{\mA}_F &= \sqrt{\trace\mA\mA^H}                           \\
             &= \sqrt{\sum_{i=1}^m \sum_{j=1}^n |\mA_{ij}|^2 }  \\
             &= \sqrt{\sum_{i=1}^m \eig(A^H A)_i }
\end{align}

\subsubsection{Special Properties}
\begin{align}
\norm{\mA\vx}_2         &\le \norm{\mA}_F \norm{\vx}_2~~~\vx\in\sRn \\
\norm{\mA\mB}_F         &\le \norm{\mA}_F \norm{\mB}_F \\
\norm{\mC-\vx\vx^T}_F^2 &= \norm{\mC}_F^2+\norm{\vx}_2^4-2 \vx^T \mC \vx
\end{align}

\subsection{Operator Norms}
For $p=1,2,\infty$ or other values, an operator norm indicates the maximum input-output gain of the matrix.
\begin{equation}
\norm{\mA}_p=\max_{\norm{\vu}_p=1} \norm{\mA\vu}_p
\end{equation}

\begin{align}
\norm{\mA}_1
  &=\max_{\norm{\vu}_1=1} \norm{\mA\vu}_1       \\
  &=\max_{j=1,\ldots,n} \sum_{i=1}^m |\mA_{ij}| \\
  &=\textrm{Largest absolute column sum}
\end{align}

\begin{align}
\norm{\mA}_\infty
  &=\max_{\norm{\vu}_\infty=1} \norm{\mA\vu}_\infty  \\
  &=\max_{j=1,\ldots,m} \sum_{i=1}^n |\mA_{ij}| \\
  &=\textrm{Largest absolute row sum}
\end{align}

\begin{align}
\norm{\mA}_2
  &=\textrm{``spectral norm"}                   \\
  &=\max_{\norm{\vu}_2=1} \norm{\mA\vu}_2       \\
  &=\sqrt{\max(\eig(\mA^T\mA))} \\
  &=\textrm{Square root of largest eigenvalue of~}\mA^T\mA
\end{align}



\subsubsection{Special Properties}
\begin{align}
\norm{\mA\vu}_p &\le \norm{\mA}_p \norm{\vu}_p \\
\norm{\mA\mB}_p &\le \norm{\mA}_p \norm{\mB}_p \\
\end{align}

\subsection{Spectral Radius}
Not a proper norm.
\begin{equation}
\rho(\mA)=\textrm{spectral radius}(\mA)=\max_{i=1,\ldots,n} | \eig(\mA)_i |
\end{equation}

\subsubsection{Special Properties}
\begin{align}
\rho(\mA) &\le \norm{\mA}_p \\
\rho(\mA) &\le \min(~\norm{\mA}_1, \norm{\mA}_\infty) \\
\end{align}


\section{Vectors}

\begin{align}
\norm{\vx}_1      &= \sum_i |\vx_i| & \textrm{L1-norm\index{L1-norm}} \\
\norm{\vx}_p      &= (\sum_i |\vx_i|^p)^{1/p} & \textrm{P-norm\index{P-norm}} \\
\norm{\vx}_\infty &= \max_i |\vx_i| & \textrm{L$\infty$-norm\index{L$\infty$-norm}, L-infinity norm}
\end{align}

\subsection{Identities}

\begin{align}
2\norm{\vu}_2^2+2\norm{\vv}_2^2 &= \norm{\vu+\vv}_2^2 + \norm{\vu-\vv}_2^2 & \textrm{Polarization Identity} \\
<\vx,\vy> &= \frac{1}{4}(\norm{\vx+\vy}_2^2-\norm{\vx-\vy}_2^2)~~\forall \vx,\vy\in\mathcal{V} & \textrm{Polarization Identity}
\end{align}


\subsection{Bounds}

\begin{align}
|\vx^T \vy| &\le \norm{\vx}_2 \norm{\vy}_2 & \textrm{Cauchy-Schwartz Inequality} \\
|\vx^T \vy| &\le \sum_{k=1}^n |\vx_k \vy_k| \le \norm{\vx}_p \norm{\vx}_q~~~\forall p,q\ge1: 1/p+1/q=1 & \textrm{H\"older Inequality}
\end{align}

For $\vx\in\mathbb{R}^n$
\begin{equation}
\frac{1}{\sqrt{n}}\norm{\vx}_2
\le\norm{\vx}_\infty
\le\norm{\vx}_2
\le\norm{\vx}_1
\le\sqrt{\textrm{card}(\vx)}\norm{\vx}_2
\le\sqrt{n}\norm{\vx}_2
\le n \norm{\vx}_\infty
\end{equation}

For any $0<p<q$ we have that $\norm{\vx}_q\le\norm{\vx}_p$.



\chapter{Bounds} %TODO: Reorganize

\section{Matrix Gain}
\begin{equation}
\lambda_\textrm{min}(\mA^T\mA)\le \frac{\norm{\mA\vx}_2^2}{\norm{\vx}_2^2}\le\lambda_\textrm{max}(\mA^T\mA)
\end{equation}

\begin{equation}
\max_{\vx\ne0} \frac{\norm{\mA\vx}_2}{\norm{\vx}_2}=\norm{\mA}_2=\sqrt{\lambda_\textrm{max}(\mA^T\mA)}\implies\vx=u_1
\end{equation}

\begin{equation}
\min_{\vx\ne0} \frac{\norm{\mA\vx}_2}{\norm{\vx}_2}=\sqrt{\lambda_\textrm{min}(\mA^T\mA)}\implies\vx=u_n
\end{equation}

\section{Rayleigh quotients}
The Rayleigh quotient of $\mA\in\sSn$ is given by
\begin{equation}
\frac{\vx^T \mA \vx}{\vx^T\vx}~~\vx\ne0
\end{equation}

\begin{equation}
\lambda_\textrm{min}(\mA)\le \frac{\vx^T \mA \vx}{\vx^T\vx} \le \lambda_\textrm{max}(\mA)~~\vx\ne0
\end{equation}

\begin{align}
\lambda_\textrm{max}(A)&=\max_{\vx: \norm{\vx}_2=1} \vx^T\mA\vx=u_1 \\
\lambda_\textrm{min}(A)&=\min_{\vx: \norm{\vx}_2=1} \vx^T\mA\vx=u_n
\end{align}
where $u_1$ and $u_n$ are the eigenvectors associated with $\lambda_\textrm{max}$ and $\lambda_\textrm{min}$, respectively.







\chapter{Equations}

\section{Linear Equations}
The linear equation $\mA\vx=\vy$ with $\mA\in\sRmn$ admits a solution iff $\rank([\mA \vy])=\rank(\mA)$. If this is satisfied, the set of all solutions is an affine set $\mathcal{S}=\{\vx=\bar \vx+z: z\in\ns(\mA)\}$ where $\bar \vx$ is any vector such that $\mA\bar\vx=\vy$. The solution is unique if $\ns(\mA)=\{0\}$.

$\mA\vx=\vy$ is \textit{overdetermined} if it is tall/skinny ($m>n$); that is, if there are more equations than unknowns. If $\rank(\mA)=n$ then $\dim\ns(\mA)=0$, so there is either no solution or one solution. Overdetermined systems often have no solution ($\vy\notin\range(\mA)$), so an approximate solution is necessary. See \autoref{sec:least-squares}.

$\mA\vx=\vy$ is \textit{underdetermined} if it is short/wide ($n>m$); that is, if has more unknowns than equations. If $\rank(\mA)=m$ then $\range(\mA)=\sRm$, so $\dim\ns(\mA)=n-m>0$, so the set of solutions is infinite. Therefore, finding a single solution that optimizes some quantity is of interest.

$\mA\vx=\vy$ is \textit{square} if $n=m$. If $\mA$ is invertible, then the equations have the unique solution $\vx=\mA^{-1}\vy$. See \autoref{sec:minimum-norm}.

\section{Least-Squares}
\label{sec:least-squares}
For an overdetermined system we wish to find:
\begin{equation}
\min_\vx \norm{\mA\vx-\vy}_2^2
\end{equation}
Since $\mA\vx\in\range(\mA)$, we need a point $\tilde \vy = \mA\vx^*\in\range(\mA)$ closest to $\vy$. This point lies in the nullspace of $\mA^T$, so we have $\mA^T(\vy-\mA\vx^*)=0$. There is always a solution to this problem and, if $\rank(\mA)=n$, it is unique~\citep[p.\ 161]{Calafiore2014}
\begin{equation}
\vx^*=(\mA^T\mA)^{-1}\mA^T\vy
\end{equation} %TODO: Check

\subsection{Regularized least-squares with low-rank data}

For $\mA\in\sRmn$, $\vy\in\sRm$, $\lambda\ge0$, the regularized least-squares problem
\begin{equation}
\textrm{argmin}_\vx \norm{\mA\vx-\vy}_2^2 + \lambda\norm{\vx}_2^2
\end{equation}
has a closed form solution
\begin{equation}
\label{equ:regularized_least_squares}
\vx = (\mA^T\mA   + \lambda \mI)^{-1}\mA^T\vy
\end{equation}
However, if $\mA$ has a $\rank{r}\ll\min(n,m)$ and a known low-rank decomposition $\mA=\mL\mR^T$ with $\mL\in\mathbb{R}^{m,r}$ and $\mR\in\mathbb{R}^{n,r}$, then we can rewrite \autoref{equ:regularized_least_squares} as
\begin{equation}
\vx = (\mR^T \mR\mL^T \mL   + \lambda \mI)^{-1}\mL^T\vy
\end{equation}
This decreases the time complexity from $O(mn^2 + n^\omega)$ to $O(nr^2+mr^2)$.

\section{Minimum Norm Solutions}
\label{sec:minimum-norm}
For undertermined systems in which $\mA\in\sRmn$ with $m<n$. We wish to find
\begin{equation}
\min_{\vx: \mA\vx=\vy} \norm{\vx}_2
\end{equation}
The solution $\vx^*$ must be orthogonal to $\ns(\mA)$, so $\vx^*\in\range(\mA^T)$, so $\vx^*=\mA^Tc$ for some $c$. Substituting into $\mA\vx=\vy$ gives $\mA \mA^T c=\vy$, therefore~\citep[p.\ 162]{Calafiore2014}:
\begin{equation}
\vx^*=\mA^T(\mA\mA^T)^{-1}\vy
\end{equation}

\section{The Sylvester Equation: $\mA\mX+\mX^T\mB=\mC$}
The equation
\begin{equation}
\mA\mX+\mX^T\mB=\mC
\end{equation}
is called a T-Sylvester equation, or *-Sylvester equation in the complex case. It can be solved using methods from, e.g.:~\citet{Teran2011,Teran2019,Dopico2016}.


\chapter{Updates}

%TODO
% \section{Low-Rank Updates to $Ax=b$}

% Given nonsingular $\mA\in\sRnn$ and $\vu,\vv\in\sRn$ with $1+\vv^T\mA^{-1}\vu\ne0$, if we have solved $\mA\vx=\vb$, then we can quickly find $(\mA+\vu\vv^T)\bar \vx = \vb$. Namely,

\section{Woodbury Identity (rank-$k$ update to inverse)}
\label{sec:update:woodbury}

%TODO: Add material from "ON DERIVING THE INVERSE OF A SUM OF MATRICES"
%TODO: Add material from "Generalization of the matrix inversion lemma"

The inverse of a rank-$k$ update of some matrix $\mA$ can be computed by doing a rank-$k$ update of $\mAi$.

\begin{align}
%(\mA + \mC\mB\mC^T)^{-1} = \mAi - \mAi\mC(\mBi + \mC^T\mAi\mC)^{-1}\mC^T\mAi %TODO: What is this good for?
(\mA+\mU\mC\mV)^{-1} &= \mAi - \mAi\mU(\mBi+\mV\mAi\mU)^{-1}\mV\mAi \\
\end{align} %TODO
where $\mA\in\sRnn$, $\mC\in\sRkk$, $\mU\in\sRnk$, $\mV\in\sRkn$, and $\mA$ and $\mC$ non-singular.

If $\mU$ and $\mV$ are vectors, then the Woodbury Identity reduces to the Sherman--Morrison formula (\autoref{sec:update:sherman}).

If $\mP,\mR$ are positive definite and $\mP\in\sRnn$, $\mR\in\sRkk$, and $\mB\in\sRkn$, then
\begin{align}
(\mPi + \mB^T\mRi\mB)^{-1} &= \mP - \mP\mB^T(\mB\mP\mB^T+\mR)^{-1}\mB\mP  \eqcite{WellingXXXX} \\
(\mPi + \mB^T\mRi\mB)^{-1} \mB^T\mRi &= \mP\mB^T(\mB\mP\mB^T+\mR)^{-1}    \eqcite{WellingXXXX}
\end{align}

\section{Sherman--Morrison Formula (rank-1 update to inverse)}
\label{sec:update:sherman}
The inverse of a rank-1 update of some matrix $\mA$ can be computed by doing a rank-1 update of $\mAi$.
\begin{equation}
(\mA + \vu\vv^T)^{-1}=\mAi-\frac{\mAi\vu\vv^T\mAi}{1+\vv^T\mAi\vu}
\end{equation}
This is a special case of the Woodbury Identity (\autoref{sec:update:woodbury}).


\section{Removing a row from $\mA^T\mA$ ($\mA^T\mA\rightarrow \mA_{\bs i}^T\mA_{\bs i}$)}

\textbf{Plain English:} Matrix times its transpose after eliminating row $i$ from the matrix

\textbf{Inputs:} $\mA\in\sR^{k,m},\vu\in\sRm,\vv\in\sRn$ and $i$, the row to remove from $\mA$

\textbf{Reduces to:} $\mC\in\sR^{k,l}$

\textbf{Algorithm:}

% Recall that
% \begin{equation}
% (\mA\mB)_{kl} = \sum_m \mA_{km}\mB_{ml}~~~\mA\in\sR^{k,m},\mB\in\sR^{m,l}
% \end{equation}
% then we have that
% \begin{equation}
% (\mA^T\mA)_{kl} = \sum_{m} \mA_{mk}\mA_{ml}=\sum_{m\ne i} \mA_{mk}\mA_{ml} + \mA_{jk}\mA_{jl}=\sum_{m\ne i} \mA_{mk}\mA_{ml} + (\mA_{k*})_{j} (\mA_{l*})_{j}
% \end{equation}
% where $(\mA_k*)_{j}$ is the $j$th element of the $k$th column of $\mA$. %TODO

% Thus,
\begin{align}
\mA_{\bs i}^T\mA_{\bs i} &= \mA^T\mA-\mA_{*i}\mA_{*i}^T \\
\noalign{Similarly:}
\mA_{\bs i}^T  y_{\bs i} &= \mA^Ty  -\mA_{*i}y_i^T
\end{align}


\section{$\mathbf{1}_r^T \mA \mathbf{1}_c$}

\textbf{Plain English:} The sum of the elements of the matrix.

\textbf{Reduces to:} Scalar

\textbf{Notation:} For $\mA \in \mathbb{R}^{r\times c}$, $\mathbf{1}_r$ is in $\mathbb{R}^{r \times 1}$ and $\mathbf{1}_c$ is in $\mathbb{R}^{c \times 1}$.

\textbf{Algorithm:} Traverse all the element of the matrix in order keeping track of the sum. For applications where accuracy is important and the matrices have a large dynamic range, Kahan summation or a similar technique should be used.

\textbf{Update Algorithm:} If an entry changes, subtract its old value from the sum and add its new value to the sum.

\section{$\mathbf{e}_i \mA \mathbf{e}_j$}

\textbf{Plain English:} Extract element $\mA_{ij}$ from the matrix

\textbf{Reduces to:} Scalar

\textbf{Notation:} TODO

\textbf{Algorithm:} TODO

\textbf{Update Algorithm:} TODO


\section{$\vx^T \mA \vx$}

\textbf{Plain English:} TODO

\textbf{Reduces to}: Scalar

\textbf{Notation:} $\mA$ must be in $\mathbb{R}^{i\times i}$. $\vx$ is in $\mathbb{R}^{i \times 1}$.

\textbf{Algorithm:} TODO

\textbf{Update Algorithm:} We make use of the identity $(\vx^T \mA \vx)=\sum_{i,j}\big((\vx \vx^T)\circ\mA\big)$. If an entry $\mA_{i,j}$ in the matrix changes subtract its old value $\vx_i \vx_j \mA_{ij}$ and add the new value $\vx_i \vx_j \mA_{ij}'$. If an entry $\vx_i$ changes TODO.


\chapter{Optimization}

\section{Standard Forms}

\textbf{Least Squares}
\begin{equation}
\min_{\vx\in\sRn} \norm{\vy-\mA\vx}_2
\end{equation}

\textbf{LASSO}
\begin{equation}
\min_{\vb\in\sRn} \left(\frac{1}{N}\norm{\vy-\mX\vb}_2^2+\lambda\norm{\vb}_1\right)
\end{equation}

\textbf{LP: Linear program}
\begin{mini!}{\vx}{\vc^T \vx}{}{}
\addConstraint{\mA_\textrm{eq}\vx}{= \vb_\textrm{eq}}
\addConstraint{\mA\vx}{\le \vb}
\end{mini!}

\textbf{Linear Fractional Program}
\begin{maxi!}{\vx}{\frac{\vc^T\vx + a}{\vd^T \vx + b}}{}{}
\addConstraint{\mA\vx}{\le \vb}
\end{maxi!}
Additional constraints must ensure $\vd^T \vx + b$ has the same sign throughout the entire feasible region.


\textbf{QCQP: Quadratic Constrainted Quadratic Programs}
\begin{mini!}{\vx}{\vx^T\mH_0\vx+2\vc_0^T\vx + \vd_0}{}{}
\addConstraint{\vx^T\mH_i\vx+2\vc_i^T\vx + \vd_i}{\le 0}{~~i\in\mathcal{I}}
\addConstraint{\vx^T\mH_j\vx+2\vc_j^T\vx + \vd_j}{  = 0}{~~j\in\mathcal{E}}
\end{mini!}
If $\mH_i\ispsd 0~\forall i$, then the program is convex. In general, QCQPs are NP-Hard.


\textbf{QP: Quadratic Program}
\begin{mini!}{\vx}{\frac{1}{2}\vx^T\mH_0\vx+\vc_0^T\vx}{}{}
\addConstraint{\mA_\textrm{eq}\vx}{=\vb_\textrm{eq}}
\addConstraint{\mA\vx}{\le \vb}
\end{mini!}
If $\mH_0\ispd 0$, then the program is convex.

If only equality constraints are present, then the solution is the linear system:
\begin{equation}
\begin{bmatrix}
\mH_0 & \mA^T \\
\mA & 0
\end{bmatrix}
\begin{bmatrix} \vx \\ \lambda \end{bmatrix}
=\begin{bmatrix} -\vc_0 \\ \vb \end{bmatrix}
\end{equation}
where $\lambda$ is a set of Lagrange multipliers.

For $\mH_0\ispd 0$, the ellipsoid method solves the problem in polynomial time.~\citep{Kozlov1980} If, $\mH_0$ is indefinite, then the problem is NP-hard~\citep{Sahni1974}, even if $\mH_0$ has only one negative eigenvalue~\citep{Pardalos1991}.

\textbf{SOCP: Second Order Cone Program (Standard Form)}
\begin{align}
\min_{\vx}      ~& \vc^T \vx \\
\textrm{s.t.}   ~& \norm{\mA_i \vx+\vb_i}_2\le \vc_i^T \vx+\vd_i,~~i=1,\ldots,m
\end{align}

\textbf{SOCP: Second Order Cone Program (Conic Standard Form)}
\begin{align}
\min_{\vx}      ~& \vc^T \vx \\
\textrm{s.t.}   ~& (\mA_i \vx+\vb_i, \vc_i^T \vx+\vd_i)\in\mathcal{K}_{m_i} ~~i=1,\ldots,m
\end{align}

\section{Transformations}

\subsection{Linear-Fractional to Linear}
We transform a Linear-Fractional Program
\begin{maxi!}{\vx}{\frac{\vc^T\vx + a}{\vd^T \vx + b}}{}{}
\addConstraint{\mA\vx}{\le \vb}
\end{maxi!}
where $\vd^T \vx + b$ has the same sign throughout the entire feasible region to a linear program using the Charnes--Cooper transformation~\citep{Charnes1962} by defining
\begin{align}
\vy &= \frac{1}{\vd^T\vx+b}\cdot\vx \\
t   &= \frac{1}{\vd^T\vx+b}
\end{align}
to form the equivalent program
\begin{maxi!}{\vy,t}{\vc^T\vy + at}{}{}
\addConstraint{\mA\vy}{\le \vb t}
\addConstraint{\vd^T\vy+bt}{=1}
\addConstraint{t}{\ge0}
\end{maxi!}
We then have $\vx^*=\frac{1}{t}\vy$.

\subsection{LP as SOCP}

The linear program
\begin{mini!}{\vx}{\vc^T \vx}{}{}
\addConstraint{\mA\vx}{\le \vb}
\end{mini!}
becomes can be cast as an SOCP:
\begin{mini!}{\vx}{\vc^T \vx}{}{}
\addConstraint{\norm{\mC_i \vx+\vd_i}_2}{\le \vb_i - \va_i^T \vx}{\forall i}
\end{mini!}
where $\mC_i=0, d_i=0~\forall i$.

\subsection{QCQP as SOCP}

The quadratic constrainted quadratic program
\begin{mini!}{\vx}{\vx^T\mQ_0\vx+\va_0^T\vx}{}{}
\addConstraint{\vx^T\mQ_i\vx+\va_i^T\vx}{\le b_i}{~~i=1,\ldots,m}
\end{mini!}
with $\mQ_i=\mQ_i^T\ispsd0$, $i=0,\ldots,m$ can be cast as an SOCP:
\begin{mini!}{\vx,t}{\va_0^T\vx + t}{}{}
\addConstraint{\norm{\begin{bmatrix} 2 \mQ_0^{1/2}\vx \\ t-1 \end{bmatrix}}_2}{\le t+1}
\addConstraint{\norm{\begin{bmatrix} 2 \mQ_i^{1/2}\vx \\ b_i-\va_i^T\vx-1 \end{bmatrix}}_2}{\le b_i-\va_i^T\vx+1}{~~i=1,\ldots,m}
\end{mini!}


\subsection{QP as SOCP}

The quadratic program
\begin{mini!}{\vx}{\frac{1}{2}\vx^T\mQ\vx+\vc^T\vx}{}{}
\addConstraint{\va_i^T\vx}{\le \vb_i}
\end{mini!}
with $\mQ=\mQ^T\ispsd0$ can be cast as an SOCP:
\begin{mini!}{\vx,\vy}{\vc^T \vx+y}{}{}
\addConstraint{\norm{
\begin{bmatrix} 2 \mQ^{1/2}\vx \\ y-1 \end{bmatrix}}_2}{\le y+1}
\addConstraint{\va_i^T \vx}{\le \vb_i}{~~\forall i}
\end{mini!}

\subsection{Sum of L2 Norms to SOCP}

\begin{mini!}{\vx}{\sum_{i=1}^p \norm{\mA_i\vx-\vb_i}_2}{}{}
\end{mini!}
becomes
\begin{mini!}{\vx,y}{\sum_{i=1}^p y_i}{}{}
\addConstraint{\norm{\mA_i\vx-\vb_i}_2}{\le y_i}{~~i=1,\ldots,p}
\end{mini!}

\subsection{Minimax of L2 Norms to SOCP}

\begin{mini!}{\vx}{\max_{i=1,\ldots,p} \norm{\mA_i\vx-\vb_i}_2}{}{}
\end{mini!}
becomes
\begin{mini!}{\vx,y}{y}{}{}
\addConstraint{\norm{\mA_i\vx-\vb_i}_2}{\le y}{~~i=1,\ldots,p}
\end{mini!}

\subsection{Hyperbolic Constraints to SOCP}

For scalar $w$, a constraint of the form
\begin{equation}
w^2\le xy, ~~x\ge0, ~~y\ge0
\end{equation}
can be transformed into the SOCP constraint
\begin{equation}
\norm{\begin{bmatrix} 2w \\ x-y \end{bmatrix}}_2 \le x+y \eqcite{Lobo1998}
\end{equation}

For vector $\vw$, a constraint of the form
\begin{equation}
\vw^T\vw = \norm{\vw}_2^2 \le xy, ~~x\ge0, ~~y\ge0
\end{equation}
can be transformed into the SOCP constraint
\begin{equation}
\label{equ:hyperbolic_constraint_to_socp}
\norm{\begin{bmatrix} 2\vw \\ x-y \end{bmatrix}}_2 \le x+y \eqcite{Lobo1998,Alizadeh2003}
\end{equation}
Note that this implies that
\begin{equation}
x^{-1}\le y \iff \norm{\begin{bmatrix} 2 \\ x-y \end{bmatrix}}_2 \le x+y %TODO: For x>0 ?
\end{equation}

%TODO: From slides
% A constraint of the form
% \begin{equation}
% \norm{x}_2^2\le 2yz, ~~y\ge0, ~~z\ge0
% \end{equation}
% can be transformed into the SOCP constraint
% \begin{equation}
% \norm{\begin{bmatrix} x \\ \frac{1}{\sqrt{2}}(y-z) \end{bmatrix}}_2 \le \frac{1}{\sqrt{2}}(y+z)
% \end{equation}

%TODO Lobo1998 fractional constraints as SOCPs

\subsection{Matrix Fractional to SOCP}

The problem
\begin{mini!}{\vx}{(\mF\vx+\vg)^T(\mP_0+\vx_1\mP+\ldots+\vx_p\mP_P)^{-1}(\mF\vx+\vg)}{}{}
\addConstraint{\mP_0+\vx_1\mP+\ldots+\vx_p\mP_P}{>0}
\addConstraint{\vx}{\ge0}
\end{mini!}
where $\mP_i=\mP_i^T\in\sRnn$, $\mF\in\sRnp$, $\vg\in\sRn$, and $\vx\in\sRp$ can be transformed into the SOCP where $t_i\in\sR, \vy_i\in\sRn$:
\begin{mini!}{\vx,t}{t_0+\ldots+t_p}{}{}
\addConstraint{\mP_0^{1/2}\vy_0+\ldots+\mP_p^{1/2}\vy_p}{=\mF\vx+\vg} \eqcite{Lobo1998}
\addConstraint{\norm{\begin{bmatrix} 2 \vy_0 \\ t_0-1\end{bmatrix}}_2}{\le t_0+1}
\addConstraint{\norm{\begin{bmatrix} 2 \vy_i \\ t_i-x_i \end{bmatrix}}_2}{\le t_i+x_i}{~~i=1,\ldots,p}
\end{mini!}

\subsection{Fractional Objective to SOCP}

Convert
\begin{mini!}{\vx}{\frac{f(x)^2}{g(x)}}{}{}
\addConstraint{g(x)}{>0}
\end{mini!}
to
\begin{mini!}{\vx,t}{t}{}{}
\addConstraint{f(x)^2}{\le t g(y)}
\addConstraint{g(y)}{>0}
\addConstraint{t}{\ge 0}
\end{mini!}
and apply \autoref{equ:hyperbolic_constraint_to_socp}.

\subsection{Chance-Constrained LP to SOCP}

The problem
\begin{mini!}{\vx}{\vc^T \vx}{}{}
\addConstraint{\textrm{Prob}\{\va_i^T\vx\le \vb_i\}}{\ge p_i}{~~i=1,\ldots,m}
\end{mini!}
where $p_i>0.5$ and all $\va_i$ are independent normal random vectors with expected values $\bar \va_i$ and covariance matrices $\Sigma_i\ispd0$, can be transformed into the SOCP:
\begin{mini!}{\vx}{\vc^T \vx}{}{}
\addConstraint{\bar \va_i^T \vx \le b_i-\Phi^{-1}(p_i)\norm{\Sigma_i^{1/2}\vx}_2}{~~i=1,\ldots,m}
\end{mini!}
where $\Phi^{-1}(p)$ is the inverse cumulative probability distribution of a standard normal variable.

%https://stanford.edu/class/ee364a/lectures/chance_constr.pdf
Likewise, the problem
\begin{maxi!}{\vx}{\vc^T \vx}{}{}
  \addConstraint{\textrm{Prob}\{\va_i^T\vx\le \vb_i\}}{\le p_i}{~~i=1,\ldots,m}
\end{maxi!}
transforms to
\begin{maxi!}{\vx}{\vc^T \vx}{}{}
  \addConstraint{\bar \va_i^T \vx \ge \Phi^{-1}(1-p_i)\norm{\Sigma_i^{1/2}\vx}_2}{~~i=1,\ldots,m}
\end{maxi!}
provided $p_i\le0.5$.

\subsection{Robust LP with Box Uncertainty as LP}

The problem
\begin{mini!}{\vx}{\vc^T \vx}{}{}
\addConstraint{\va_i^T \vx}{\le b_i}{~~\forall \va_i\in\{\hat \va_i + \rho_i \vu : \norm{\vu}_\infty\le1\}}{~~i=1,\ldots,m}
\end{mini!}
is equivalent to
\begin{mini!}{\vx}{\vc^T \vx}{}{}
\addConstraint{\hat \va_i^T \vx + \rho_i\norm{\vx}_1}{\le b_i}{~~i=1,\ldots,m}
\end{mini!}
which is equivalent to:
\begin{mini!}{\vx}{\vc^T \vx}{}{}
\addConstraint{\hat \va_i^T \vx + \rho_i \sum_{j=1}^n \vu_j}{\le b_i}{~~i=1,\ldots,m}
\addConstraint{-\vu_j}{\le \vx_j\le\vu_j}{~~j=1,\ldots,n}
\end{mini!}

\subsection{Robust LP with Ellipsoidal Uncertainty as SOCP}

The problem
\begin{mini!}{\vx}{\vc^T \vx}{}{}
\addConstraint{\va_i^T \vx}{\le b_i}{~~\forall \va_i\in\{\hat \va_i + \mR_i \vu : \norm{\vu}_2\le1\}}{~~i=1,\ldots,m}
\end{mini!}
is equivalent to
\begin{mini!}{\vx}{\vc^T \vx}{}{}
\addConstraint{\hat \va_i^T \vx + \norm{\mR_i^T \vx}_2}{\le b_i}{~~i=1,\ldots,m}
\end{mini!}

\subsection{Square Root as SOCP}
\begin{equation}
\sqrt{x}\ge t \iff x\ge t^2 \iff \norm{\begin{matrix} 1-x \\ 2t \end{matrix}}_2 \le 1+x %TODO: For x>=0?
\end{equation}

The problem
\begin{mini!}{\vx}{\vc^T \vx}{}{}
\addConstraint{\va_i^T \vx}{\le b_i}{~~\forall \va_i\in\{\hat \va_i + \mR_i \vu : \norm{\vu}_2\le1\}}{~~i=1,\ldots,m}
\end{mini!}
is equivalent to
\begin{mini!}{\vx}{\vc^T \vx}{}{}
\addConstraint{\hat \va_i^T \vx + \norm{\mR_i^T \vx}_2}{\le b_i}{~~i=1,\ldots,m}
\end{mini!}


\section{Useful Problems}

\begin{align}
\textrm{average}(\vv) &= \min_{x\in\sR} \norm{\vv-x\mathbf{1}}_2^2 \\
\textrm{median}(\vv) &= \min_{x\in\sR} \norm{\vv-x\mathbf{1}}_1
\end{align}

\chapter{Algorithmics}

\section{Time Complexities}
\begin{center}
{\footnotesize\renewcommand{\arraystretch}{1.2}
\begin{tabular}{p{1.5cm}p{3cm}p{3cm}p{4cm}p{1cm}}
\textbf{Operation}             & \textbf{Input}                   & \textbf{Output}    & \textbf{Algorithm}                     & \textbf{Time}   \\ \hline
Matmult
    & $A,B\in n\times n$
    & $n \times n$
    & Schoolbook
    & $O(n^3)$
    \\ \hline

    &
    &
    & Strassen~\citep{Strassen1969}
    & $O(n^{2.807})$
    \\ \hline

    &
    &
    & Best
    & $O(n^\omega)$
    \\ \hline
Matmult
    & $A\in n\times m, B\in m\times p$
    & $n \times p$
    & Schoolbook
    & $O(nmp)$
    \\ \hline
Inversion
    & $A\in n\times n$
    & $n \times n$
    & Gauss--Jordan elimination
    & $O(n^3)$
    \\ \hline

    &
    &
    & Strassen~\citep{Strassen1969}
    & $O(n^{2.807})$
    \\ \hline

    &
    &
    & Best
    & $O(n^\omega)$
    \\ \hline
SVD
    & $A\in m\times n$
    & $m\times m, m\times n, n\times n$ \newline $m\times r, r\times r, n\times r$
    &
    & $O(mn^2)$ \newline \hbox{$(m\ge n)$} \\ \hline
Determinant
    & $A\in n\times n$
    & Scalar
    & Laplace expansion
    & $O(n!)$         \\ \hline

    &
    &
    & Division-free~\citep{Rote2001}
    & $O(n!)$
    \\ \hline

    &
    &
    & LU decomposition
    & $O(n^3)$
    \\ \hline

    &
    &
    & Integer preserving~\citep{Bareiss1968}
    & $O(n^3)$
    \\ \hline
Back \newline substitution
    & $A$ triangular
    & $n$ solutions
    & Back substitution
    & $O(n^2)$
    \\ \hline
\end{tabular}
}
\end{center}

\subsubsection{A comment on $\omega$}

The lower bound on matmult time complexity is $O(n^\omega)$, where $\omega$ is an unknown constant bounded by $2\le\omega\le2.3728596$ (\autoref{tbl:omega-vals} lists the known upper bound on $\omega$ over time). Algorithms achieving lower values of $\omega$ tend to be less efficient in practice for all but the largest matrices. Of the algorithms with times of less than $O(n^3)$, only the Strassen algorithm~\citep{Strassen1969} has seen serious attempts at optimized implementation. Most matmult implementations use highly optimized variants of the standard $O(n^3)$ algorithm. At this point, memory and bus speeds dominate the performance of implementations, so simple Big-O notation cannot be used to reliably compare matmult performances.

The time complexity for solving sparse linear systems was bounded by $\omega$ until recently, when randomized methods were used to obtain a bound of $O(n^{2.331645})$~\citep{Peng2021}.

\begin{table}
\centering
\begin{tabular}{lll}
\textbf{Name}           & \textbf{Year} & $\omega$  \\ \hline
Standard                & -             & 3         \\
\citet{Strassen1969}    & 1969          & 2.807     \\
\citet{Pan1978}         & 1978          & 2.796     \\
\citet{Bini1979}        & 1979          & 2.78      \\
\citet{Schonhage1981}   & 1981          & 2.548     \\
\citet{Schonhage1981}   & 1981          & 2.522     \\
\citet{Romani1982}      & 1982          & 2.517     \\
\citet{Coppersmith1982} & 1982          & 2.496     \\
\citet{Strassen1986}    & 1986          & 2.479     \\
\citet{Copper1990}      & 1990          & 2.376     \\
\citet{Williams2012}    & 2012          & 2.37293   \\
\citet{Williams2012}    & 2012          & 2.37287\footnote{The original conference paper gave a bound of $\omega<2.3727$, but omitted some necessary constraints. This corrected value appears in the final paper.}   \\
\citet{LeGall2014}      & 2014          & 2.3728642 \\
\citet{LeGall2014}      & 2014          & 2.3728640 \\
\citet{LeGall2014}      & 2014          & 2.3728639 \\
\citet{Alman2020}       & 2020          & 2.3728596
\end{tabular}
\caption{Upper bounds on the value of $\omega$ over time \label{tbl:omega-vals}}
\end{table}


%\section{Gram-Schmidt}
%TODO
% Consider the [[Gram–Schmidt process]] applied to the columns of the full column rank matrix <math>A=[\mathbf{a}_1, \cdots, \mathbf{a}_n]</math>, with [[inner product]] <math>\langle\mathbf{v},\mathbf{w}\rangle = \mathbf{v}^\top \mathbf{w}</math> (or <math>\langle\mathbf{v},\mathbf{w}\rangle = \mathbf{v}^* \mathbf{w}</math> for the complex case).

% Define the [[Vector projection|projection]]:
% :<math>\mathrm{proj}_{\mathbf{u}}\mathbf{a}
% = \frac{\left\langle\mathbf{u},\mathbf{a}\right\rangle}{\left\langle\mathbf{u},\mathbf{u}\right\rangle}{\mathbf{u}}
% </math>
% then:
% :<math>
% \begin{align}
%  \mathbf{u}_1 &= \mathbf{a}_1,
%   & \mathbf{e}_1 &= {\mathbf{u}_1 \over \|\mathbf{u}_1\|} \\
%  \mathbf{u}_2 &= \mathbf{a}_2-\mathrm{proj}_{\mathbf{u}_1}\,\mathbf{a}_2,
%   & \mathbf{e}_2 &= {\mathbf{u}_2 \over \|\mathbf{u}_2\|} \\
%  \mathbf{u}_3 &= \mathbf{a}_3-\mathrm{proj}_{\mathbf{u}_1}\,\mathbf{a}_3-\mathrm{proj}_{\mathbf{u}_2}\,\mathbf{a}_3,
%   & \mathbf{e}_3 &= {\mathbf{u}_3 \over \|\mathbf{u}_3\|} \\
%  & \vdots &&\vdots \\
%  \mathbf{u}_k &= \mathbf{a}_k-\sum_{j=1}^{k-1}\mathrm{proj}_{\mathbf{u}_j}\,\mathbf{a}_k,
%   &\mathbf{e}_k &= {\mathbf{u}_k\over\|\mathbf{u}_k\|}
% \end{align}
% </math>

% We can now express the <math>\mathbf{a}_i</math>s over our newly computed orthonormal basis:

% :<math>
% \begin{align}
%  \mathbf{a}_1 &= \langle\mathbf{e}_1,\mathbf{a}_1 \rangle \mathbf{e}_1  \\
%  \mathbf{a}_2 &= \langle\mathbf{e}_1,\mathbf{a}_2 \rangle \mathbf{e}_1
%   + \langle\mathbf{e}_2,\mathbf{a}_2 \rangle \mathbf{e}_2 \\
%  \mathbf{a}_3 &= \langle\mathbf{e}_1,\mathbf{a}_3 \rangle \mathbf{e}_1
%   + \langle\mathbf{e}_2,\mathbf{a}_3 \rangle \mathbf{e}_2
%   + \langle\mathbf{e}_3,\mathbf{a}_3 \rangle \mathbf{e}_3 \\
%  &\vdots \\
%  \mathbf{a}_k &= \sum_{j=1}^{k} \langle \mathbf{e}_j, \mathbf{a}_k \rangle \mathbf{e}_j
% \end{align}
% </math>
% where <math>\langle\mathbf{e}_i,\mathbf{a}_i \rangle = \|\mathbf{u}_i\|</math>. This can be written in matrix form:
% :<math> A = Q R </math>
% where:
% :<math>Q = \left[ \mathbf{e}_1, \cdots, \mathbf{e}_n\right] </math>
% and
% :<math>
% R = \begin{pmatrix}
% \langle\mathbf{e}_1,\mathbf{a}_1\rangle & \langle\mathbf{e}_1,\mathbf{a}_2\rangle &  \langle\mathbf{e}_1,\mathbf{a}_3\rangle  & \ldots \\
% 0                & \langle\mathbf{e}_2,\mathbf{a}_2\rangle                        &  \langle\mathbf{e}_2,\mathbf{a}_3\rangle  & \ldots \\
% 0                & 0                                       & \langle\mathbf{e}_3,\mathbf{a}_3\rangle                          & \ldots \\
% \vdots           & \vdots                                  & \vdots                                    & \ddots \end{pmatrix}.</math>



\bibliography{refs}

\printindex

\end{document}

TODO:
Orthogonal matrix = all eigenvalues are 1 or -1

Centering matrix
Distance matrix



For two vectors $b$ and $x$, $p=\frac{b^Tx}{b^Tb}b$ is the projection of $x$ onto $b$.

TODO: Gilber Strang (2016, p563: Matrix Factorizations)

TODO: Strang 2016, p. 583, List of symbols and computer codes


TODO: Add Gram-Schmidt procedure
TODO: Add computational efficiency notes for QR


Highlighting matrix example: http://www.texample.net/tikz/examples/highlighting-matrix/